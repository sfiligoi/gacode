\section{Source Terms }\label{source}

\subsection{Beam Related Quantities}

At the present time \ot relies on the Callen \cite{Callen74:1} model of fast ion
slowing down to generate beam related results. An initial modification of the
fast ion deposition profile to account for prompt orbit smearing  is optional.
Transient effects associated with beam turn on and off are accounted for as
explained in section \ref{tbeam}. 

Let the index m range over all possible beam energies (and beam lines).
Then the beam related terms are defined as follows:
\begin{description}
 \item[$hibrz_m$] The normalized fast ion birth rate. This is the basic quantity
 determined by Freya.  Hibrz is the  normalized birth distribution of fast ions
 per unit volume per unit time and is determined by
 \beq
  hibrz_m= \frac{\frac{n_z}{V_z}}{\frac{n_t}{V_p}}
 \eeq
 where $\frac{n_z}{n_t} $ is the fraction of ions that make it
 into the plasma and are born in poloidal flux zone z, $V_z$
 and $V_p $ are the flux zone and total plasma volume respectively.
 Since $n_t$ is given by
 \beq
  n_t=\sum_{all zones}n_z
 \eeq
 we see that the volume average birth rate density is unity:
 \beq
  \frac{1}{V_P}\sum_{all zones}hibrz_m V_z =
  \bigg(\frac{1}{n_t}\bigg)\sum_{all zones}n_z =1
 \eeq
 Typically hibrz has a value greater than 1 near the magnetic axes indicating
 beam penetration  better than average .
 \item[$hdepz_m$] Prompt orbit averaged version of hibrz (optional).
 \item[$hdep_m(\rho)$]  Either the birth or orbit averaged
 deposition (hbirz or hdepz depending on what the
 user selected) interpolated from the zone to the 
 transport grid.
 \item[$bke_m$] Toroidal momentum fraction transferred to electrons.
 Identical to $K_e$ in \cite{Callen74:1}.
 \item[$bki_m$] Toroidal momentum fraction transferred to ions 
 Identical to $K_i$ in \cite{Callen74:1}.
 \item[$fbe_m$] The fraction of the initial beam energy that is
 deposited on electrons during the slowing down process.
 Identical to $G_e$ in \cite{Callen74:1}.
 \item[$fbi_m$] The fraction of the initial beam energy that is deposited on
 ions during the slowing down process. Identical to $G_i$ in \cite{Callen74:1}.
 Both $G_e $ and $G_i $ are  based on a fast ion distribution function which
 allows for charge exchange during the slowing down process. However further
 re-ionization of the neutral created by the fast ion is neglected. $fbe,\ fbi$
 are output.
 \item[$pbeam_m$] The beam power (input)
 \item[$xloss_m$] Fraction of beam power lost due to aperture, orbit, and shine
 through.
 \item[$qb(\frac{w}{cm^3})$] Beam power deposited in the plasma:
 \beq
  qb =\sum_m qb_m
 \eeq
 \beq
  qb_m(\rho)   = (1.0-xloss_m)*pbeam_m*hdep_m(\rho)/V_p
  \label{eq:qbm}
 \eeq
 ($qb_m$ is printed out in the beam tables)
 \item[$spbr\frac{g}{(cm sec^2)}$] Toroidal angular momentum source.
 \beq
  spbr_m(\rho)   = angmpf_m(\rho)*sb_m(rho)
 \eeq
 \item[$sb_m$] The source density of fast ions; it is defined in terms of the
 beam energy, $e_m$, and $q_b$ as:
 \beq
 sb_m(\rho)=qb_m(rho)/e_m
 \eeq
 \item [$ angmpf_m (\frac{g.cm^2}{sec})$] The average momentum of a single beam
 ion born in a flux zone. The number of flux zones (in poloidal flux space)is
 controlled by the input value \emph{mf}. Let $n_\zeta$ be the number of ions
 born in zone $\zeta$. The ions have toroidal speed $v_I $ and major radius $R_i
 $ at the birth point. The average angular momentum of an ion in the zone is
 then given by
 \beq
  \overline{mvR}=\frac{1}{n_\zeta} \sum_{i \in \zeta}m_bv_i R_i
 \eeq
 and $angmpf_m $ is this quantity interpolated onto the $\rho $ grid, for beam
 energy component \emph{m}.
\end{description}
Other beam related quantities required below are:
\begin{description}
 \item[$\tau_s$] The
 Spitzer momentum exchange time for electron-ion collisions
 \beq
  \tau_s= \left (\frac{3}{4\sqrt{2\pi}}\right )\frac{\sqrt{m_e T_e^3}}
  {Z_f^2 n_e q_e^4\left [24.-\ln\left (\frac{\sqrt{n_e}}{T_e}\right )\right ]} 
  \label{eq:taus}.
 \eeq
 \item[$\tau_f$] The fast ion lifetime, defined in terms of the critical speed,
 $v_c$ (speed at which the fast ion slowing down rate on electrons and ions is
 equal) and the initial fast ion speed, $v_b$:
 \beq
  \tau_f= \frac{\tau_s}{3}\ln\bigg({1+\frac{1}
  {(\frac{v_c}{v_b})^3}}\bigg) .
 \eeq
 The fast ion lifetime against charge exchange is evaluated at the beam energy
 (per atomic mass unit) relative to the rotating ion distribution,
 $\frac{e_{rel}}{atw_f}$ :
 \beq
  \tau_{cx}= \frac{1}{n_n \Xi}.
 %  \sigma_{cx}v(\frac{e_{rel}}{atw_f})}
  \label{eq:taucx}
 \eeq
 $n_n$ is the total neutral density, and the reaction rate $\Xi
 \frac{cm^3}{sec}$ is a function of the relative speed between the beam ions and
 the thermal neutrals. The original version of \ot uses
 \[
  \Xi = \langle \sigma_{cx} v \rangle
 \]
 evaluated at the energy $\frac{e_{rel}}{atw_f} $, where $e_{rel} $ is the
 relative energy accounting for bulk rotation but not thermal motion of the
 neutrals. Furthermore the neutrals are assumed to have the same bulk rotation
 as the ions in this expression.  A new optional form for $\Xi $ is also
 available:
 \[
  \Xi = \sigma(e_{rel})v_{rel}(rtstcx).
 \]
 The factor $rtstcx $ is intended to allow some scoping of the sensitivity of
 the fast ion distribution  function to  the charge exchange form assumed.
 Setting $rtstcx < -20.$ causes the code to use the average fast ion speed in
 determination of $\Xi $. At this time vrel neglects any thermal motion of
 neutrals as above.  The Maxwellian charge exchange rate, $\langle \sigma_{cx}v
 \rangle $, and the cross section $\sigma_{cx} $  are both  taken from Ref.
 \cite{Freeman:1974}. The code assumes $\sigma_{cx}v=0$ if the energy is greater
 than 100 keV/amu.  
 \item[$fbth_m$] represents the fraction of the  fast ion population that
 thermalizes. (This leads to a source of thermal energy as well as thermal
 ions.) \ot assumes that the ratio $\frac{\tau_s}{\tau_{cx}} $ is  independent
 of the fast ion speed. Consequently the fraction of the fast ions  that
 thermalize without charge exchange becomes
 \beq
  fbth_m=exp(-\frac{\tau_f}{\tau_{cx}}).
 \eeq
 Note that the rate at which fast ions are  lost from the system due to charge
 exchange with thermal neutrals is consequently
 \beq
  sscxl_m =(1-fbth_m)sb_m.
 \eeq
 Re-ionization of these fast neutrals is neglected. sscxl is neglected in the
 particle source term $S_i$ of Eqs.~\eqref{eq:ni} and \eqref{eq:neut}, but is
 included in the energy term qcx; see below. 
\end{description}

\subsubsection{Transient Beam Effects}\label{tbeam}

\ot evolves the beam related quantities based on the rate equation with a
constant source $s_l$ in time interval $\delta t_l$:
\beq
 \pdiff {x}{t} +\frac{x}{\tau_l}=s_l.
\eeq
Here $x$ stands for any of a number of fast ion quantities identified in this
document as having ben aged (?). The analytic solution of this equation at the
end of the time interval $\delta t_l$, given an initial condition $x_0$ at the
start, is
\beq
 x=x_0\exp \left (-\frac{\delta t_l}{\tau_l} \right ) +
 s_l\tau_l \left ( 1. -\exp \left (-\frac{\delta t_l}{\tau_l}
 \right ) \right )
\eeq
In the code the time step $\delta t_l $ is governed by the
predictor/corrector solution scheme (see section ? ) and is
equal to the time interval $\theta \delta t $ during the course
of the solution. (At the start special adjustments are made as
discussed below.) During each time interval the source of fast ions, $s_l$, is either assumed equal to the source in the previous
time interval
or is replaced with a new source because a new beam deposition
calculation was carried out, based on plasma conditions at the
central time $t+\theta \delta t $. (No allowance for changing the
beam power is made; $ \tau_l$ is also evaluated at that time.)

%--------------------------------------------------------------------------
\subsection{Particle Sources}

\begin{description}
 \item[$S_i$] is the source density for ions of species i in \Eqref{eq:ni}.
 \beq
  S_i=sbcx+scx+sbeam+sfusion+sion+srecom
 \eeq
 \begin{description}  % match ni
  \item[sbcx $ \frac{\#}{cm^3sec} $] A source of fast ions (and also thermal
  neutrals with energy $\frac{3}{2}T $) due to charge exchange of beam  neutrals
  with thermal ions. sbcx is derived either from hbirz (no prompt orbit
  averaging) or hdepz (with prompt orbit averaging):
  \begin{eqnarray}
   sbcx_i(\rho)=\sum_m sb_m hicm_m^i
  \end{eqnarray}
  $hicm_m^i$ is the fraction of the fast ion birth rate for beam  component m
  that leads to neutrals of type i.
  \item [scx $\frac{\#}{cm^3sec} $] charge exchange between thermal ions and
  thermal neutrals. If two neutral species, corresponding to two primary ion
  species are present, then scx will be a sink for one species and a source for
  the other. That is, ion species a charge exchanges with neutral species b
  producing a neutral of type a and an ion of type b. The charge exchange rate,
  $ cx12r(\rho)$, is based on the Freeman-Jones\cite{Freeman:1974} cross
  sections. For ion density $n_i $ and  neutral density $n_{nj}$, where $nj$ is
  the other species, we have:
  \begin{eqnarray}
   scx_i(\rho)= n_i(\rho)n_{nj}(\rho)cx12r(\rho)
  \end{eqnarray}
  If only one neutral species is present then there is no particle source, $
  scx=0$ . (There is an energy source however because the ion and neutral
  temperatures are not assumed equal).
  \item[sbeam $\frac{\#}{cm^3sec}$]  Source of thermal ions due to beam slowing
  down. All fast ions which do not experience charge exchange during their
  lifetime are assumed to slow down into the corresponding thermal distribution:
  \begin{eqnarray}
   sbeam_i(\rho)= \sum_mfbth_m(\rho) sb_m(\rho) 
  \end{eqnarray}
  where $sb$ and $fbth_m$ are  defined above.
  \item[sion $\frac{\#}{cm^3sec}$] Source of ions due to electron impact
  ionization of neutrals:
  \beq
   sion=\sum_{i=1}^2 n_{n_i}(\rho) * eirate(\rho)
  \eeq
  where $eirate$ is the electron impact ionization rate of
  of atomic hydrogen taken from \cite{Freeman:1974} 
  \beq
   eirate \equiv n_e \langle \sigma v \rangle
  \eeq
 \end{description} %end match ni
 \item[$S_i^{2D}$] represents a source term due to MHD evolution and is given by
 \beq
  S_i^{2D}=-n_i\ddiff{t} \ln H + \frac{d}{H}\ddiff{\rho}Hn_i
 \eeq
 This term is included in $sother $ in the table labeled ``particle sources''.
\end{description}

%--------------------------------------------------------------------------
\subsection{Energy Sources}

\subsubsection{Electrons}

The energy sources and sinks ($\frac{keV}{cm^3sec} $) for  \Eqref{eq:eenergy}
are as follows:
\begin{description}
 \item[$Q_e$] is given by
 \begin{multline}
  Q_e= -qexch + qohm -qrad-qione+qbeame \\
  +qrfe-qpe+qfuse
 \end{multline}
 \begin{description} %begin 0a
  \item [qexch] represents the electron-ion energy exchange term
  \item [qohm] represents ohmic heating
  \item [qrad] represents radiative losses
  \item [qione] represents an energy sink due to recombination
  \item [qbeame] represents electron energy source due to neutral beam heating.
  Let m be a multi-index that ranges over beam lines and beam energy components
  (i.e. full, half, and third). Then we may write
  \begin{multline}
   qbeame_m(\rho)=fbe_m(\rho)\cdot qb_m(\rho)+bke_m(\rho)\cdot
   spbr_m(\rho)\cdot\omega(\rho) \label{eq:qbeamem}
  \end{multline}
  \beq
   qbeame(\rho)=\sum_m qbeame_m.
  \eeq
  \item [qrfe] represents electron energy source due to rf heating. qrfe can be
  input, obtained from simple models, or dynamic coupling of Onetwo to Toray for
  ECH heating and current drive is available.
  \item [qpe] represents energy loss due to pellet ablation
  \item [qfuse] represents electron heating due to fusion.
 \end{description}  %end 0a
 \item[$\omega L_e $]
 \beq
  \omega L_e =\omega * sprbeame
 \eeq 
 \item [$S_{T_e}^{2D}$] represents heating of electrons due to evolution of the
 mhd equilibrium.
 \beq
  S_{T_e}^{2D}=-\frac{5}{2}n_eT_e\ddiff{t}\ln H 
  +\left (\pdiff{ \ln \rho}{t} \right )
   \bigg [ \frac{5}{2}n_eT_e \pdiff {\ln H} {\rho} \\
    +\frac{3}{2}T_e\sum_{i=1}^{nion}Z_I\pdiff{n_i}{\rho} 
    +\frac{3}{2} \left (n_e+T_e\sum_{i=1}^{nion} n_z\pdiff{Z_i}{T_e}
    \right )\pdiff{T_e}{\rho} \bigg]
 \eeq
 In the output of \ot the definition
 \beq
  qe2d\equiv S_{T_e}^{2D}
 \eeq
 is used.
\end{description}    


%-------------------------------------------------------------------------
\subsubsection{Ions}

The source terms ($\frac{kev}{cm^3sec}$) for the ion energy equation,
\Eqref{eq:ionenergy}, are defined as follows:
\begin{description}
 \item [Q ] The term Q appearing on the rhs of the ion energy 
 equation is defined as
 \beq
 Q(\rho)=qexch+ qioni-qcx+qbeami+qfusi+qrfi
 \eeq
 \begin{description}  %begin a
  \item [qexch$\frac{keV}{cm^3sec}$] is the electron ion energy exchange term
  due to Coulomb collisions. (See above for definition.)
  \item [qioni] is defined as electron impact ionization of neutrals minus
  thermal ion recombination :
  \begin{multline}
   qioni = \frac{3}{2}\sum_{i=1}^{nprim} \bigg ( 
    sion_i(\rho)T_{n_i} (\rho)
    - srecomb_i(\rho) T \bigg )
  \end{multline}
  \item[qcx ] qcx is the compound term  defined as 
  \begin{multline}
   q_{cx}(\rho)=\sum_{i=1}^2\bigg (\overbrace{1.5T(\rho)*sbcx_i(\rho)}^a \\
   \overbrace{ -1.5*T_{n_i}(\rho)*sscxl_i(\rho)*ibcx}^b \bigg)\\ 
   \overbrace{ +1.5*n_{n_i}(\rho)*cext_i(\rho)*n_i(\rho)*
   (T(\rho)-T_{n_i}(\rho))}^c \\
   \overbrace{+1.5n_i n_{n_k} r_l*cxr(\frac{T}{atw_i})*
    (T(\rho)-T_{n_k}(\rho))}^d
  \end{multline}
  \begin{description}
   \item{a} represents LOSS of average energy $\frac{3}{2}T(Kev)$  per ion due
   to charge exchange with beam neutral
   \item{b} represents source of energy due to fast ion thermal neutral charge
   exchange (ibcx=0 or 1, user selectable), $sscxl_i$ is the fraction of sscxl
   that leads to species i ions, and $T_{n_i}$ is the temperature of neutral
   species i.
   \item{c} represents charge exchange with thermal neutrals of the same species
   \item{d} represents charge exchange with neutral species k, $r_l$ is a
   correction factor for plasma elongation.
  \end{description}
  \item[qbeami] represents heating of ions by the beam.
  \begin{multline}
   qbeami_m(\rho)=fbi_m(\rho)*qb_m(\rho)+bki_m(\rho)*
   spbr_m(\rho)*\omega(\rho)\\
   +fbth_m*sb_m(\rho)*erot \label{eq:qbeamim}
  \end{multline}  
  \beq
   qbeami(\rho)=\sum_m qbeami_m
  \eeq  
  \item[qfusi]  heating of ions due to fusion.
  \item[qrfi] represents the rf ion heating term . Qrfi can be  directly input
  into the code, some simple models are available internally in onetwo and
  dynamic coupling to the  ray tracing code Curray is also available.
 \end{description}   %end a
 \item [$S_T^\omega (\frac{keV}{cm^3sec})$] Represents the source  of kinetic
 rotational energy which is the sum of four terms:
 \beq
  S_T^\omega = sprcxe+sprcxree+spreimpe + \omega L_e
 \eeq
 \begin{description}  %begin b
  \item[sprcxe] Gives the  source of rotational kinetic energy due to thermal
  charge exchange . If only a single neutral species is present then the sum
  involving k below is absent. For the two neutral (and hence also ion) case
  charge exchange can  occur with a neutral of different mass as well as
  different momentum as given by the second sum:
  \begin{multline}
   sprcxe=\sum_{i=1}^{2} \frac{\langle R\rangle}{\langle R^2\rangle} 
    n_{n_i}cexr_in_im_i(v_n^2-v_z^2)\\
   %    +\sum \begin{Sb}  %note: Sb environment is a amstex thingy,
   %             i=1 \\   %doesnt work with amsmath !!
   %           k=3-i \\
   %           \end{Sb}
   %          \begin{Sp}
   %           2 \\
   %        \end{Sp} 
   +\sum_{\substack{i=1 \\ k = 3-i}}^{2}
   \bigg(\frac{\langle R\rangle}{\langle R^2\rangle}\bigg)^2 
    n_i cx12 n_k (m_kv_n^2-m_iv_z^2)
  \end{multline}
  \item[sprcxree ] is the ion rotational kinetic energy source due to
  recombination; it is given in terms of sprcxre [\Eqref{eq:sprcxre}]
  \beq
   sprcxree= -\frac{1}{2}\omega* sprcxre
  \eeq
  \item[spreimpe] Gives the source of rotational kinetic energy due to electron
  impact ionization of thermal neutrals. It is defined in terms of spreimpt
  [\Eqref{eq:spreimpt}]:
  \beq
   spreimpe(\rho)=\frac{1}{2}\omega_n *spreimpt
  \eeq
 \end{description}  %end b
 \item[$S_T^{2D}$ ] represents ion thermal energy sources due to time evolving
 MHD equilibria.
 \begin{multline}
  S_T^{2D}= -\frac{5}{2}T\pdiff {\ln H}{t} \sum_{i=1}^{nion} n_i
  +\pdiff{\ln \rho}{t}\bigg (  
   \frac{5}{2}T\pdiff{\ln H}{\rho}\sum_{i=1}^{nion} n_i \\
   +\frac{3}{2}T\sum_{i=1}^{nion}\pdiff{n_i}{\rho}
   +\frac{3}{2}\pdiff{T}{\rho}\sum_{i=1}^{nion} n_i
  \bigg )
 \end{multline}
 \item [$S_T^{2D\omega}$ ] represents ion rotational kinetic energy sources due
 to time evolving MHD equilibria.
 \begin{multline}
  S_T^{2D\omega}= 
  -\frac{1}{2}angrm2d(1)\langle R^2\rangle\omega^2\pdiff{\ln H}{t}
   \sum_{i=1}^{nion}n_i m_i \\
  +\frac{1}{2}angrm2d(2)\omega\left[spr2d +\omega 
  \left(\pdiff{\langle R^2\rangle}{t}+\langle R^2\rangle\pdiff{\ln H}{t}\right)
  \sum_{i=1}^{nion}m_in_i \right] \\
  -\frac{1}{2}amgrm2d(3)\omega^2\pdiff{\langle R^2\rangle}{t}
  \sum_{i=1}^{nion}n_im_i
 \end{multline}
 The sum of the the evolving MHD related energy sources is called qi2d in the
 output of \ot:
 \beq
  qi2d \equiv S_T^{2D}+S_T^{2D\omega} \label{eq:qi2d}
 \eeq
 The $ angrm2d(1,2,3)$  multipliers are user selectable input to \ot, defaulted
 to 1.0
\end{description}

%-----------------------------------------------------------------------------
\subsection{Toroidal Momentum Sources}

All source terms below are in units of  $[\frac{g}{cm \cdot sec^2}]$. The
electrons are assumed to have negligible momentum (i.e. no separate equation for
the electron toroidal momentum is introduced; however some momentum sources
associated with the fast ion-electron interactions are included below as ion
terms). At the present time sources and sinks associated with \emph{ion impact
ionization} are neglected.
\begin{description}
 \item[ $S_\omega $] is defined as
 \begin{multline}
  S_\omega=Sprbeame+Sprbeami+Sprcxl+Spreimpt+\\
  Sprcx+Sprcxre 
 \end{multline}
 where the definition of each of the individual terms follows.
 \begin{description}  %begin 1b
  \item[sprbeame] is the (delayed) source of angular momentum  transferred from
  the (beam) fast ions  to the electrons during the slowing down process. This
  angular momentum is rapidly shared with the ions and is thus a source term for
  the thermal ion toroidal momentum equation:
  \beq
   sprbeame(\rho)=bke(\rho,e_b,b_j)*spbr(\rho,e_b,b_j)
  \eeq
  \item[sprbeami] is the (delayed) source of angular momentum transferred from
  the (beam) fast ions to the thermal ion fluid.
  \begin{multline}
   sprbeami(\rho)=bki(\rho,e_b,b_j)*spbr(\rho,e_b,b_j)+ \\
   fbth(\rho,e_b,b_j)*sb(\rho,e_b,b_j)*atw_b \\
   *m_p*v_z(\rho)\frac{\langle R^2\rangle}{\langle R\rangle}
  \end{multline}
  \item[ssprcxl] represents the gain of  angular momentum due to charge exchange
  of a fast ion with a thermal neutral (the thermal neutral adds its momentum to
  the thermal ion distribution)
  \begin{multline}
   ssprcxl(\rho)=fprscxl*spbr(\rho,e_b,b_j) \\
   +fscxl*sb(\rho,e_b,b_j)*atw_b \\
   *m_p*v_z(\rho)*\frac{\langle R^2\rangle}{\langle R\rangle}
  \end{multline}
  \item[spreimpt] represents gain of momentum due to electron impact ionization
  of thermal neutrals
  \begin{multline}
   spreimpt(\rho)=\sum_{i=1}^{nprim}eirate(\rho)\cdot enn_i(\rho)
   m_i\cdot vneut_i(\rho)\cdot\langle R\rangle \label{eq:spreimpt}
  \end{multline}
  \item[sprcx] represents the sorce/sink of momentum due to charge exchange of a
  thermal neutral with a thermal ion. For two ion and neutral species we have
  \begin{multline}
   sprcx(\rho)=\\ \sum_{i=1}^{nprim}enn_i(\rho)\cdot cexr_i(\rho)\cdot
   en_i(\rho)\cdot atw_i\cdot m_p\cdot\langle R\rangle \cdot(vneut_i(\rho)-vionz(\rho))\\
   +cxmix\cdot(atw_k\cdot vneut_k(\rho)-atw_i\cdot v_z(\rho))
  \end{multline}
  \item[sprcxre] represents the sink of thermal angular momentum due to charge
  exchange of a thermal ion with a fast neutral and also includes radiative
  recombination of thermal ions.
  \beq
   sprcxre(\rho)=\sum_{i=1}^{nprim}sbcx_i(\rho)*\langle R^2\rangle
   \omega m_i  \label{eq:sprcxre}
  \eeq
 \end{description} %end 1b
 \item[$S_\omega^{2D}$] is the source term due to evolution of the MHD
 equilibrium. It is given by
 \begin{multline}
  S_\omega^{2D}=-\omega\sum_{i=1}^{nion}\bigg( m_in_i\ddiff{t}\langle R^2\rangle
  +m_in_i\langle R^2\rangle\ddiff{t}\ln H \bigg )\\
  +\frac{d}{H}\ddiff{\rho}\bigg(
  H\omega\sum_{i=1}^nion m_in_i\langle R^2\rangle \bigg )
 \end{multline}
 spr2d  is the name of this term in the code output. At present the individual
 contributions are not broken out. In the code this term  is multiplied by an
 input factor $angrm2d(4)$,which is defaulted to 1.0 but the user can assign any
 value (eq. 0.0) to gauge the effect of this term.
\end{description}

\subsection{Other Definitions used in Onetwo Output}

\subsubsection{Flux Tables}

\begin{description}
 \item[angmtm] The total flux associated with toroidal rotation
 (see  \Eqref{eq:omgam}) is given in
 outone in the table labeled fluxes, under the column headed angmtm.
\end{description}
In the table labeled Energy Fluxes (ref. the ion energy equation,
\Eqref{eq:ionenergy}): (flxangce  is  The energy flux due to particle
convection
\begin{description}
 \item[omegapi] $\frac{keV}{cm^2 sec}$ The energy flux associated  with
 ``conduction'' (really viscosity) omegapi is defined as 
 \beq
  omegapi\equiv \omega\Pi = \omega \Gamma_\omega^{cond}
 \eeq
 See the discussion of \Eqref{eq:gwcond} above regarding how this quantity is
 obtained in analysis and simulation modes.
 \item[cvctvrot] $\frac{kev}{cm^2 sec}$ The energy flux due to
 momentum convection (flxangce internal to code), given by 
 \Eqref{eq:cvct}.
\end{description}

\subsubsection{ION Power Balance Tables}

The table labeled ``ion Energy Sources'' contains 
\beq
 qomegapi \equiv \frac{1}{H\rho}\ddiff {\rho}\bigg(
 H\rho\omega\Pi \bigg)
\eeq
\beq
 omegale  \equiv \omega L_e = \omega* sprbeame 
\eeq

The following terms are used in the table ``Ion Energy
Sources Due to Angular Rotation'' (ref. the ion 
energy equation, \Eqref{eq:ionenergy} )
\beq
 wdnidt(\rho) \equiv\sum_{i=1}^{nion} \frac{1}{2}m_i\omega^2
  \langle R^2\rangle \pdiff{n_i}{t}
\eeq
\beq
 niwdwdt(\rho)\equiv \sum_{i=1}^{nion} \frac{1}{2}m_in_i\omega 
 \langle R^2\rangle\pdiff{\omega}{t}
\eeq
\beq
 omegdgam(\rho)\equiv qomeapi+vischeat
\eeq
\beq
 vischeat(\rho)\equiv -\Pi\pdiff{\omega}{\rho}
\eeq
\beq
 qangce(\rho)\equiv \frac{1}{H\rho}\ddiff {\rho}H\rho \Gamma_T^\omega
\eeq
\beq
 thcx(\rho)\equiv sprcxe =  
\eeq
\beq
 rec+fcx(\rho)\equiv sprcxree
\eeq
\beq
 e-impact(\rho)\equiv spreimpe =
\eeq

\subsubsection{Momentum Balance Tables}

The table labeled ``Momentum Balance and Confinement Time'' gives
\beq
 qangce(\rho)\equiv 
\eeq
The table labeled ``Toroidal Rotation results'' gives

\begin{description}
 \item[flxangce] The energy flux due to particle convection, Eq[??] is called
 flxangce in the code
\end{description}
The table labeled ``Toroidal Rotation Sources'' gives the terms that appear on
the RHS of \Eqref{eq:omega}. These terms are covered above in Section 2.4.

\subsection{Neutral Beam Injection Tables}
The column labeled ``fast ion energy source'' is 
the rate at which energy is instantaneously
deposited in the plasma. This quantity is equal to $qb_m$ in the
steady state and is larger/smaller than $qb_m$ during
transient beam turn on/off (see aging of beam parameters).
The column labeled ``delayed e. source'' is the quantity
$ qb_m $, see \Eqref{eq:qbm}. (Note that $qbeame+qbeami+qcx =qb$
at all times.) $qb_m$ is equal to the instantaneous energy deposition
in the plasma in steady state. Otherwise $qb_m$ reflects the finite
buildup and decay times of the fast ion density.
The columns labeled ``energy fraction deposited in electrons
/ions'' are the quantities $fbe_m,fbi_m$ (see Beam Related
Quantities).
The column labeled ``p. slowing down time'' gives the product
$\tau_s N$ defined by Eqs.~\eqref{eq:taus} and \eqref{eq:7}.
The column labeled ``e. slowing down time'' gives the product
$\frac{1}{2}\tau_s G_e$ (see beam section).

%---------------------------------------------------------------------
\subsection {Items in the Summary Page}

\begin{description}
 \item[Beam power elec. $\equiv pbel $] The beam power delivered to the
 electrons is given by:
 \beq
  pbel = \sum_m  fpe_m \int  qb_m  dV .
 \eeq
 \item[Beam power ions. $\equiv pbion $] The beam power delivered to the ions 
 is given by:
 \beq
  pbion = \sum_m  fpi_m \int  qb_m  dV .
 \eeq
 Note that fast ion charge exchange is implicit in the terms $fpe_m$ and $fpi_m$
 [the fast ion distribution function used to generate these quantities is given
 in \Eqref{eq:1}], where the charge exchange factor, $P_{cx}$, \Eqref{eq:2}, is
 calculated  using the fixed $\tau_{cx} $ given by \Eqref{eq:taucx}. 

 %constant .
 %See definitions  under the beam section and Eq.[\ref{eq:qbm}].
 \item[Beam power cx loss $\equiv pbcx$] This value gives the beam power lost
 due to charge exchange. It is determined by what is left over after the  beam
 power delivered to the electrons and ions is accounted for:
 \beq
  pbcx = \sum_m  \int \left ( qb_m-qbeame_m-qbeami_m \right) dV .
 \eeq
 The sum is over beam components and energies; see Eqs.~\eqref{eq:qbm},
 \eqref{eq:qbeamem}, and \eqref{eq:qbeamim}.
\end{description}


