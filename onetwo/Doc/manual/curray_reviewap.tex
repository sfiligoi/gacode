\renewcommand{\textfraction}{0.10}
\renewcommand{\topfraction}{0.90}  % not greater than 1- textfraction
\renewcommand{\bottomfraction}{0.65} % not greater than 1- textfraction
\renewcommand{\floatpagefraction}{0.60}

\section{Running Curray in \ot}

 Funding provided by the National Transport Code Collaboratory 
(NTCC) enabled us to install and review the ray tracing code \ct
at General Atomics. The project consisted of building an interface to
the \ct code in \ot and subsequently testing the \ct code with L and H
mode DIII-D discharges. 

  We found that there were  two  different ways
that profile information is passed to \ct from \ot. The first method,
exemplified by the NTCC version of the code, used input from an
(undocumented) file onetwo.out.  The updated version \ct obtained
from TK Mau uses files  masanori.in, bdens and ebeam\_d to pass thermal
and fast ion profiles to \ct. Furthermore, for other transport codes (e.g.
Transp) a different file, trxpl.out, is used for this purpose.
Since the information that is passed to \ct from a transport code is
not specific to a transport code we decided to use this opportunity
to unify the transport code interface. Consequently  the interface
described below is intended to be general enough so that other transport
codes wishing to couple to \ct can use this method. Accordingly we
have incorporated the existing Transp interface (using file trxpl.out)
in our scheme. The  method of profile input using files
masanori.in, etc.  was retained (as an option) in the version of
\ct now installed at GA. We did not create a code to write files
in the format required for onetwo.out. This means that the existing
NTCC version of \ct can not currently run the DIII-D test cases
presented below. 
Note that other (non \ot), users  of the code will  not notice any change in
the interface  since it was created in such a way as to be 
transparent. 

 The interface to the \ot transport code required that
some minor changes and new features were added to the \ct code  itself. 
The changes in \ct are clearly delineated by the ``!HSJ''
sentinel  present in some sections of the code.
To eliminate confusion and for easy reference we present here an itemized
list of the changes made. 
\begin {itemize}
 \item \ct was moved to our CVS repository /c/cvsroot and
 all future changes to the code should be derived from that
 repository. As mentioned above the starting point for this repository
 was updated version of \ct obtained directly from TK Mau. 
 \item \ct  will now accept the name of an input file
 on the execution line. This allows proper manipulation of the \ct run
 while being controlled by a transport code.
 The default input file name remains  curray\_in and hence
 if the name is not supplied at the command prompt then the old
 behavior is recovered.
 \item The name of the eqdsk to read can now be supplied as
 input in the namelist data in \texttt{input\_file}. The default name
 of this eqdsk remains eqdsk.hb so that the old behavior is
 recovered if no explicit name is given in the namelist input.
 This feature was added to allow simulations where MHD evolution is
 taking place.
 \item The default eqdsk size was made 129x129 instead of 65x65  since the
 larger size is what is routinely used at DIII-D for MHD coupled transport
 calculations. The file param.f90 must be changed and the code recompiled to
 affect this change since dynamic sizing of eqdsks is not implemented at
 present. This also means that two versions of the \ct executable have to be
 maintained to accommodate the MHD grids. The transport grid passed to \ct is
 governed by namelist input parameter \texttt{nprof} and as long  as the grid
 size is less than \texttt{nprof} (= 133) no change  is required for it. 
 \item To allow investigation of  the sensitivity to various fast ion components,
 the number of ions (parameter \texttt{nions} in param.f90), was increased to 12
 to accommodate fast ion species. Each beam line and injection energy is assumed
 to be a separate fast ion species. For the typical DIII-D modeling in \ot this
 means that the two injectors and 3 energies give rise to six fast ion species.
 In addition, fast alphas can be present as well.  At the present time each fast
 species is assumed to have a density and temperature profile given by two
 thirds of the stored energy density divided by the fast ion density. This
 feature is available only by creating the \ct input files using the \ot code.
 An option in \ot will collapse all the fast ion components into one effective
 component thereby recovering the more usual mode of operation of \ct.
 \item The new \ot interface eliminates the second eqdsk file,
 eqdskex, and the spectrum input information file, raytrin.
 Both of these files were intended for use only with \ot and hence
 will not affect other users. The information that used to reside in
 raytrin is duplicated in curray\_in 
 and the information in eqdskex is duplicated in the TRANSP
 interface file trxpl.out. The meaning of the namelist switch \texttt{ioread}
 in \ct  is thus modified accordingly. \texttt{ioread=1} now means create
 raytrout only. Its previous association with raytrin is no longer
 meaningful. \ot  was changed to
 create file curray\_in (including the spectrum information) instead of
 file raytrin. File trxpl.out, also created by \ot now contains all the
 kinetic information (e.g. Te, Ti, ne, ni, nfast, etc). The format of
 trxpl.out was not  changed so that it will still work with Transp as
 well. The meaning of the \ct namelist input switch \texttt{iprof} was
 changed to accommodate the trxpl.out file input from Onetwo.
 Previously \texttt{iprof=1} meant use Transp input file trxpl.out. Now \texttt{iprof=1}
 means use file trxpl.out, no matter what the source of that file is
 (e.g. Transp or \ot).
 \item The GA version of \ct was modified to ``correctly'' handle a
  problem with EFIT type eqdsks. An unfortunate convention in these
  eqdsks causes a sign problem in the driven current. This modification
  currently applies to all eqdsks and needs to be generalized for
  NSTX and other machines.
\end{itemize}

 On the \ot side of the interface, three files, curray\_in, eqdsk, and
 trxpl.out, are created by \ot each time \ct is run out of \ot. 
   File curray\_in is the namelist input
 file, required for all types of \ct runs,
 which includes the necessary antenna spectrum information (the
 curray\_in file is documented in file curray\_input, which is part of
 the \ct distribution). The actual
 name of the curray\_in file is determined at run time by \ot. 
 File eqdsk is a standard
 EFIT type equilibrium file whose actual name is also determined  at run
 time.

 The spectrum and power density  information for \ct is assumed to be
 present entirely  in the  \ot input file inone. An example of this
 input for DIII-D is given
 in Sec.~\ref{ct_input} for 60, 83, and 117 MHz operation.  This information is
basically identical to the information normally present in file
curray\_in (for a single time slice, input power, and spectrum) and
the description of the namelist parameters in curray\_in serves as
the input description in file inone. However, the \ot input file inone
is actually a superset of the \ct data to allow for changes in time
in the spectrum and input power and to allow for several different
antenna specifications simultaneously. This is the purpose of the
extra indices that will be found in the parameters in inone (see
Sec.~\ref{ct_input}).
 Finally a file that contains the thermal and
fast ions densities and temperatures, trxpl.out, is created and then
\ct is spawned. This process may  be repeated many times as the 
transport simulation proceeds. An  option in inone allows the saving of
the curay\_in, eqdsk, and trxpl.out files so that a standalone \ct
run can be repeated at a later time. Since \ot will write the
curray\_in file it is never necessary to create that file by
hand. Instead the information should  be placed into inone directly.
 At any given
time \ot will spawn \ct repeatedly until all \ct cases have
been done. The results are accumulated internally in \ot. The \ct
models can be freely intermixed with other heating and current drive
models. For example in Sec.~\ref{ct_input} three \ct cases are defined
based on frequency, power input, etc. These three cases could  be made
active simultaneously  by specifying equal times (rfon) for which  the models
are active. Alternatively any combination of models (fast wave,
ech, etc.) is selected using this method. 

\ct creates the files currayout and raytrout. The former is
a standard \ct output file meant for user inspection. The latter
file, raytrout, is read directly by \ot.
 We note that one of the planned enhancements of \ct mentioned on
 the NTCC web site is ``Enhance code capabilities to handle more than
 one antenna  spectrum either by source modification or by a shell
 script to do multiple CURRAY calls''. The \ot interface described
 above  does this already and we expect that most other transport
 codes (eg TRANSP) also have machinery in place to account
 for multiple \ct cases.
 

\subsection{\ct Benchmarking}\label{benchm}
   We ran both the modified
 local version of \ct and the NTCC version on  the ITER and NSTX
 test cases  supplied with the NTCC \ct distribution. Additionally 
 two new DIII-D test cases  were added to broaden the range over 
 which \ct has been successfully applied.

 The first two test cases examined are the ones supplied with the
 NTCC distribution of \ct. The object of this testing was to verify
 that the GA version of \ct yields results comparable to those
 obtained with the NTCC version. The NSTX test case given on the 
 NTCC web site 
 \cite{b2}
 has to use  the adjoint method of determining current drive
 efficiency  (which required a change in the supplied input file).
 It was observed that 
 the calculations take significantly longer when this option is used.
 Since execution  time becomes an 
 issue when \ct is called repeatedly from  a transport code,
 guidelines should be evolved to automatically determine when
 the added complication of the adjoint equations is required and how
 often the tables have to be recalculated during a transport
 simulation. At present the \ot interface does not take these matters
 into consideration.
 
 \begin{figure} %note: figure environment not available in slides
 \centering    
\begin{narrow}{-.50in}{0in}  
 \mbox{\epsfig{file=../curray/ITER_comp.eps,height=3in,width=6in}}
    \\[20pt]
 \mbox{\epsfig{file=../curray/NSTX_comp.eps,height=3in,width=6in}}
 \end{narrow}
 \caption{ (a) Electron and ion (dashed line)  absorbed power density for 20 MW of input
  power for the example ITER discharge. The corresponding driven current profiles are
  given in (b). Figures (c) and (d) give the same results for the NSTX
  case with 2.1 MW of injected power. The NTCC  and GA versions of
  \ct produce identical profiles for these cases. } 
  \label{Figbwav1a} 
 \end{figure}
 The results are shown in Fig.~\ref{Figbwav1a} for both the low aspect
 ratio NSTX case and the ITER case.  For ITER,  the driven current
 should be calculable using  either the Karney adjoint
 formulation of current drive efficiency or the Ehst-Karney small inverse
 aspect ratio approximation. As indicated in
 Fig.~\ref{Figbwav1a} (b) for the latter case, the total driven current
 is about 51 kA. When the calculation is repeated using the adjoint
 method this result drops to about 44 kA. This is perhaps a larger
 discrepancy than  one would expect for this machine.  As is evident from the
 figure  the results for the GA version of \ct and the
 NTCC version are indistinguishable. Hence we may confidently carry
 out further testing  using the more recent GA version only.

 As previously remarked we were not able to test DIII-D cases against
 the \ct version on the NTCC web site due to differences in the
 way \ot data is read by these two  versions of the code.
 This situation is independent of the new Onetwo interface and exists
 due to the development of curray that has taken place  since
 the NTCC version was released. To test \ct  with  DIII-D discharges an H mode
 (shot 111221) and an L mode (shot 84293) case were examined. The
 kinetic data for these two shots is given in Fig.~\ref{f1}. The 
 spectrum for 60, 83, and 117 MHZ was obtained from TK Mau (H mode)  and from
 Craig Petty (L mode). The total electron and ion absorbed power and
 current drive values for these discharges using the six lobe antenna
 spectra given in Sec.~\ref{ct_input} are summarized in Table~\ref{t1}. It was
 previously established \cite{b3} that six edge reflections lead to
 rf power absorption values greater than 90\% for high $\beta_e$
 discharges. Thus that number has become a de-facto standard. However as
 is seen in Table I for the L mode case 6 edge reflections are not
 sufficient. For the L mode shot we ran a second series of cases where
 100 edge reflections were allowed. This boosted the absorbed power to
 about 80\%. (A further increase in allowed reflections only slowly
increases the absorbed power as it asymptotically approaches 100\%.) 

 \begin{figure} %note: figure environment not available in slides
 \centering 
\begin{narrow}{-.50in}{0in}   
 \mbox{\epsfig{file=../curray/111221_kinetics.eps,height=3in,width=6in}}
\\[20pt]
 \mbox{\epsfig{file=../curray/84293_kinetics.eps,height=3in,width=6in}}
%/home/stjohn/transport_codes/onetwo_test_cases/curray_tests/onetwo_coupled/111221_tk_comp/NTCC_60/1_beam
\end{narrow}
 \caption{(a,b) Temperatures and densities for the H mode shot 111221
   at 3700 msec. The beam temperature is taken as $\frac{2}{3} $ of the
   average fast ion energy. $\beta = 3.23\%$, $B_T = 1.86 T$, $I_p = 1.18 MA$
   , $\overline{ n} = 4.03\times 10^{13} cm^{-3}$.
    (c,d)~The corresponding profiles for L mode shot 84293 at 2110
    msec. $\beta = 0.7\%$, $B_T = 2.08 T$, $I_p = 1.38 MA$, $\overline{ n} = 1.95 \times 10^{13} cm^{-3}$.}
  \label{f1}
 \end{figure}

\begin{table}
\begin{centering}
\begin{tabular}{ccccc}
\multicolumn{5}{c}{\bfseries Shot 84293} \\
\multicolumn{5}{c}{\bfseries (L mode) }  \\ 
FW MHz& $P_e ,kW $ & $P_i, kW  $ & $ I_{CD}, kA  $ & $P_a $ \\ \hline
60  & 501& 499 & 56& 41 \%  \\
60  & 491& 509 & 51& 80 \%  \\
%83  & 710/650 & 290/350&  87/80 & 0 & \\
83  & 744/607& 256/398  &  80/71 & 31\%   \\
83  & 728 & 272?   &  74 & 80\%   \\
117  &950  &50 & 111 & 29\%  \\
117  &895  &105 & 99 & 82\%  \\
\\ \hline
& \\
\multicolumn{5}{c}{\bfseries Shot 111221} \\
\multicolumn{5}{c}{\bfseries (H
  mode)} \\ 
FW MHz& $P_e, kW$ & $P_i ,kW $ & $ I_{CD}, kA $&  $P_a $ \\ \hline
60  &340/287& 660/773 & 22/16 & 95\% \\
83  & 420 & 580 & 30  & 99\% \\
117 & 395/212 & 605/788 & 30/18 & 99\% \\ \hline
\end{tabular}
\caption{Heating and Current drive results for DIII-D L and H
  mode cases. The second number for electron and ion absorbed power and
  current drive indicates results obtained using Transp
  profiles. $P_a$ is the percent of injected power absorbed. For the
L mode shot results are quoted for 6  and 100 edge reflections (with
the larger value of $P_a$ corresponding to 100 edge reflections and
the smaller value to 6 reflections).}
\label{t1}
\end{centering}
\end{table} 

The cases  shown in the table and in the figures below all used 1MW of
fast wave input power, an antenna location of 1 degree, and 66 rays in six
spectral power lobes (see Sec.~\ref{ct_input}). 
 It was found that allowing six edge reflections
in the ray tracing caused most of the power to be absorbed for the H
mode case. However as indicated in the table the absorption for
the L mode case tends to be weak unless the number of edge reflections
is increased significantly (maxref = 100). $P_e$, $P_i$, $I_{CD}$ are
all quoted on the basis of 100\% power absorption. Hence $P_e + P_i = 1
 MW $ for all cases shown.

 Some of the profiles associated with the results given in Table I are
 presented below. 
 To begin with we examine in Fig.~\ref{Figbwav2a} results obtained
 using Transp and \ot as the transport code drivers for \ct.  
 \begin{figure} %note: figure environment not available in slides
\begin{narrow}{-.50in}{0in} 
 \mbox{\epsfig{file=../curray/transp_prof.eps,height=3in,width=6in}}
%/home/stjohn/transport_codes/onetwo_test_cases/curray_tests/onetwo_coupled/111221_tk_comp/NTCC_60/fig2
%/home/stjohn/transport_codes/onetwo_test_cases/curray_tests/onetwo_coupled/111221_tk_comp/NTCC_60/fig2/matches_transp
\end{narrow}
 \caption{ (a) Electron and ion absorbed power density for 1 MW of input
  power at 60MHz for DIII-D discharge 111221 during  H mode phase. The blue curves are the \ct result when run out of  \ot. The red
  curves are the results obtained from Transp coupled to the NTCC
  version of Curray. The dashed lines represent the power density
  given to the ions.
(b)~The corresponding driven current profiles.
Shot 111221 @ 3710 msec.} 
  \label{Figbwav2a}
 \end{figure}
To generate these results the profiles of densities and temperatures
obtained from Transp were used in the curves corresponding to \ot.  
Hence we find that both the electron and ion
heating  and current drive are very similar. The small observed
differences  are most likely due to the difference
in equilibrium files used. For \ot an eqdsk of size 129 x 129 was
required. The Transp results use a 65 x 65 eqdsk.
 As is shown below
in practice, due to
different transport grids, and especially treatment of the fast ions,
the difference between Transp and \ot simulations will be larger. 

 The actual heating and current drive profiles determined by using \ot
 derived fast ion density and effective temperature profiles for the
 case given in 
 Fig.~\ref{Figbwav2a} is  shown in Fig.~\ref{Figbwav3a}. 
 \begin{figure} %note: figure environment not available in slides
 \centering 
 \begin{narrow}{-.50in}{0in}   
 \mbox{\epsfig{file=../curray/60MHZ_6b.eps,height=3in,width=6in}}
 \\[20pt]
 \mbox{\epsfig{file=../curray/60_83MHZ_1b_comp.eps,height=3in,width=6in}}
%/home/stjohn/transport_codes/onetwo_test_cases/curray_tests/onetwo_coupled/111221_tk_comp/NTCC_60/1_beam
 \end{narrow}
 \caption{(a,b) The heating and current drive profiles from \ct obtained
   using \ot. Both effective single beam and multiple beam results are
   shown.
    (c,d)~Comparison of the 60 and 83 MHz fast wave heating and current
    drive profiles. The electron heating has increased
    while the resonance absorption of the (fast) ions has moved closer to
    the magnetic axis and become somewhat less efficient.}
  \label{Figbwav3a}
 \end{figure}
 In addition to the usual
 single effective beam model this figure also shows the results of
 treating each individual beam component  as a separate species. Since
 each beam component will have a different fast ion stored energy
 distribution the effect on \ct was of interest. As shown in the
 figure however the results are only slightly  different when the
 multiple beam model is used. Due to the non-linear nature of the
 dispersion relation that \ct must solve we find that there is a 10\%
 difference in electron heating and current drive but only about a 5\%
 difference in ion heating (which includes the fast ions). In parts
 (c) and (d) of the figure a comparison of fast wave injection at 60
 and 83 MHz is made. The higher injection frequency favors electrons
 at the expense of ions and increases the driven current.  Of
 particular interest in Fig.~\ref{Figbwav3a}(c) is the change in the ion
 heating profile in moving from 60 to 83 MHz fast wave injection.
 For both frequencies most of the absorption in the ion channel is
 due to the fast ion contribution (at 60 MHz we have 543 out of 660 KW
 of ion heating due to the fast ions and at 83 MHz the corresponding
 numbers are 492 out of 580 KW ). The harmonics that contribute to the
 results are shown in Fig.~\ref{h1}. 
 \begin{figure} %note: figure environment not available in slides
 \centering 
\begin{narrow}{-.50in}{0in}   
 \mbox{\epsfig{file=../curray/harmonic.eps,height=3in,width=6in}}
%/home/stjohn/transport_codes/onetwo_test_cases/curray_tests/onetwo_coupled/111221_tk_comp/NTCC_60/1_beam
 \end{narrow}
 \caption{(a) Major radius location of hydrogen and  deuterium
   resonances for H mode shot 111221.03700  at 60MHz. 
    (b)~The same result for 83MHz. The magnetic axis is at 1.778 m}
  \label{h1}
 \end{figure}
 At 83 MHz, it is the 6'th harmonic of
 deuterium at $\rho =0.06$ (high field side) that causes the high $(\sim 0.79
 W/cm^3)$ absorption peak. A smaller  secondary peak at $\rho = ~0.5$
 is due to a combination of 5'th (high field) and 7'th (low field)
 harmonics for this case.  At 60MHz the 4'th harmonic of D at $\rho =
 0.26$ (high field side) leads to the sharp peak in the fast ion
 heating profile [see Fig.~\ref{Figbwav3a}(a,b)]. A second peak at $\rho =
 0.43$ is due to the 5'th harmonic of D (low field side).  This
 explains the broad absorption profile compared to the 83 MHz case as
 the waves must sample larger volumes of the plasma in order to
 encounter both the low and high field side resonances. For the 83MHz
 case the high fast ion beam temperature near the axis is
 more than able to overcome this effect. In both cases there
 is little qualitative change in the electron heating profile due to
 Landau damping and TTMP effects. 
 
  As was shown in Fig.~\ref{Figbwav2a}, the Transp and \ot results track
   closely when identical profiles are used. In Fig.~\ref{Figbwav3b} we
   show results when this is not the case. 
 \begin{figure} %note: figure environment not available in slides
 \centering 
\begin{narrow}{-.50in}{0in}   
 \mbox{\epsfig{file=../curray/12_transp_com_Hm_Hmode_117MHz.eps,height=3in,width=6in}}
\\[20pt]
 \mbox{\epsfig{file=../curray/111221_pfast.eps,height=3in,width=6in}}
 %\mbox{\epsfig{file=../curray/jFW_r_cur_tr_nc_84293_83_n.eps,height=4in,width=4in}}
\end{narrow}
 \caption{Results from \ot (blue) for the H mode discharge at 117 MHz compared with
   Transp results (red). As explained in the text the larger ion power absorption in Transp drives the observed differences.}
  \label{Figbwav3b}
 \end{figure}   
   Here the fast ion 
   distribution from Transp is sufficiently different that the ion
  heating and current drive are significantly different. It is seen in
  the figure that \ot puts more energy into the electrons and less into
  the ions. This also increases the driven electron current. The ion
  absorption in \ot is less due to the lower effective temperature of
  the fast ions species. In Fig.~\ref{Figbwav3b}(c) the thermal ion (D and
  minority species H) and fast (D)  ion absorption
  for the two cases is shown. Most of the ion absorbed power in fact
  appears in the fast ion species [dashed lines, Fig. \ref{Figbwav3b}(a)]
  inside of $\rho = 0.2$. As seen in the figure the primary and minority
  ion species contribute relatively little. In Fig.~\ref{Figbwav3b}(d) the
  effective fast ion pressure is shown for the two cases. The
  difference inside of $\rho =0.2$ is substantial and is responsible for
  the higher ion absorbed power (788 KW) obtained using the Transp
  derived profiles compared to the \ot result (605 KW). Due to the
  relatively small volume over which the effective fast ion
  temperature is significantly different in the two codes, the overall
  fast ion energy content is almost the same. Hence we see that the
  \ct results can be quite sensitive to details of the fast ion
  distribution if fast wave absorption is localized.   

 \begin{figure} %note: figure environment not available in slides
 \centering  
\begin{narrow}{-0.40in}{0in}  
 \mbox{\epsfig{file=../curray/84293_60MHz.eps,height=3in,width=6in}}
 %\mbox{\epsfig{file=harmonic_84293_60_83.eps,height=3in,width=6in}}
 %\mbox{\epsfig{file=jFW_r_cur_tr_nc_84293_83_n.eps,height=4in,width=4in}}
 \caption{(a,b) The heating and current drive results for the  L mode
   discharge 84293.2110  at 60 MHz. In (a) the solid line represents
   the electron heating and the dashed curve is the total ion heating.
   In (b) the electron driven current is shown.}
  \label{l16}
\end{narrow}
 \end{figure}
\begin{figure} %note: figure environment not available in slides
 \centering  
\begin{narrow}{-0.40in}{0in}  
 %\mbox{\epsfig{file=84293_60MHz.eps,height=3in,width=6in}}
 \mbox{\epsfig{file=../curray/harmonic_84293_60_83.eps,height=3in,width=6in}}
 %\mbox{\epsfig{file=jFW_r_cur_tr_nc_84293_83_n.eps,height=4in,width=4in}}
 \caption{(a,b) Major radius location of hydrogen and deuterium
   resonances for L mode shot 84293.02110 at 60 and 83 MHz. }
  \label{l17}
\end{narrow}
 \end{figure}
  Our second DIII-D test case uses the L mode discharge, shot
  84293@2110 msec.  For 60MHz fast
  wave injection  the results using \ot as the driver for \ct are shown
  in Fig.~\ref{l16}. %l16
  The corresponding resonance surfaces for this frequency as well as
  for 83 MHz are given in Fig.~\ref{l17}. Examination of Fig.~\ref{l17}(a) shows that
  there is a strong fourth harmonic resonance (4D) for deuterium  near the magnetic
  axis at 60 MHz injection. This causes the sharp peak in fast ion
  absorption observed near $\rho = 0.16 $ in Fig.~\ref{l16}(a). The 3D and 5D
  contributions are also visible as small peaks at $\rho = .3$ and $
  0.45 $ respectively.

 \begin{figure}
 \centering   
\begin{narrow}{-0.40in}{0in}
 %\mbox{\epsfig{file=qrfs_r_cur_tr_nc_84293_83_n.eps,height=4in,width=4in}}
 %\mbox{\epsfig{file=jFW_r_cur_tr_nc_84293_83_n.eps,height=4in,width=4in}}
 \mbox{\epsfig{file= ../curray/12_transp-84293.eps,height=3in,width=6in}}
\end{narrow}
 \mbox{\epsfig{file=../curray/beam_pressure_84293b01_b01.eps,height=4in,width=4in}}
 \caption{L mode discharge results for 83 MHz, fast wave coupled
   input power. Both the Transp and \ot results are given for the
   heating and current drive profiles. Transp has significantly more
   power absorption by the fast ions. The total and thermal pressures
  determined by Transp and \ot are given in the lower figure. } 
  \label{Figbwav3d}
 \end{figure}
   Our final example consists of the L mode discharge at 83
   MHz. In Fig.~\ref{Figbwav3d}, the results between Transp and \ot
   driven \ct are compared. The situation is quite similar to the
   H mode case shown in Fig.~\ref{Figbwav3b}. Due to the larger
   stored energy density for fast ions near the magnetic axis \ct 
   predicts higher ion heating and lower electron heating using the
   Transp Monte Carlo derived beam profiles. The 5D beam ion
   resonance near $\rho = 0.07 $ is clearly visible but the 
   electron heating due to Landau damping and TTMP effects is quite
   significant and leads to a higher driven current than was observed
   in the previous cases. The decrease in electron power absorption
   near the magnetic axis, clearly visible in Fig.~\ref{l16}(a) has also been
   observed in full wave calculations \cite{b1}. But the  full wave
   calculations are preliminary in nature so we  can not include them 
   in this report. 

    When this shot is run at 117MHz the results are as indicated in 
   Fig.~\ref{Figbwav3e}.  In this figure we have also included the
  heating and current drive profiles allowing for a maximum of 100
  reflections of the wave. As is seen in the figure most of the power
  is absorbed by the electrons which increases the driven current.
   Even though  the wave direction can change significantly
   with 100 reflections the current drive decreased only by about 10\%.
 \begin{figure}
 \centering   
\begin{narrow}{-0.40in}{0in}
 \mbox{\epsfig{file= ../curray/117-6-100-84293.eps,height=3in,width=6in}}
\end{narrow}
 \caption{Results for the L mode shot with \ot used as the driver for
   \ct. The red curves correspond to allowing 100 reflections of the
   rays.  The blue curves are for 6 reflections.}
  \label{Figbwav3e}
 \end{figure}



\subsection{Standards, problems and recommendations} \label{prob}

\ct follows  the recommended standards given in \cite{b4}.
 In particular we note that the necessary source code together with
 drivers for two test cases with input and output documentation
 are currently available.    We found that the code can be built under
   OSF F90 5.4, HP UX11.0 f90, and Linux (RH7.2,8.0, Pgf90). 
   However \ct will fail with
   a floating point error on the OSF machines for some  cases  which
   work under Linux and HPUX. Further investigation will be required to
   determine the cause of this problem.  
The Makefile was modified to allow building of
the code on all local platforms. We  did not change the basic structure of
the Makefile that allows the build to continue even when errors have
occurred. However we find this approach makes building the code more
obscure  than it needs to be since errors encountered during the
build do not terminate the build process. This  forces the user to
backtrack  through many obscure 
lines of output generated by the makefile to find out where the
error occurred. It is much easier to work through the build process 
one error at a time until the build succeeds. One possible workaround
is to at least indicate to the user  how the makefile may be used to
write to std error piped to a disk file. Then working through the
std error file from the top down will achieve the same result.
 The unused ''customization '' files present in the distribution
 should be incorporated into the mainline code
 (and then selected through input switches) or removed from the distribution altogether. Otherwise the concept
 of a single tested version of \ct becomes less obvious.
 We recommend that 
 version  numbers be attached to the code so that reference to a
 given version is uniform across all users of the code.

 A readme file (curray\_readme) together with CurrayDoc.pdf and
 curray\_input  provides  the necessary documentation. 
 As is required, no graphics is embedded in \ct . A  file suitable
 for graphics generation is written (rayop)  and may be processed using
 an open source graphics library (pgplot) together with the driver
 curplot. The functionality of curplot was found to be adequate but no
 extensive testing was undertaken. 
  

    In building and running \ct we encountered some minor problems.
    Interpretation of the validity of  results and proper input
    settings for the code also initially caused us some difficulty.
    Below  we detail some of these issues and offer
    remedies as appropriate. 
  

  Using the  files masanori.in, bdens
  and ebeam\_d is awkward and error prone in our opinion.  The reading
  of these files is initiated by the switches \texttt{gamene =0} and
 \texttt{iprof =0}. This is confusing since \texttt{iprof =0} implies that 
 analytic profiles will be read but then \texttt{gamene.ne.0} is used to
 read in tables of profiles from the above files instead.  
  The GA version of the
  code now allows this input method to be used  optionally but we
  recommend that the new \ot interface described above is used in the
  future. This will require a new identification scheme for labeling
  fast ions in file currayout. The file curray\_input which documents the namelist input
 parameters in curray\_in should be added to the NTCC distribution.

 \ct currently has a nice f90 style to it. However the modules
  are using a mixture of f90 and f77 syntax. Not all compilers have
  option flags to accept such a mixture. Since modules are an f90
  construct it would appear to us that using f90 only syntax in
  modules makes sense. We think that this should be a standard
 that NTCC contributed codes should try to achieve for sake of cross
 platform compatibility.

  We find that the recompilation of the code for different eqdsk sizes
  is awkward. Although this feature is present in most codes that we
  know about, \ct is in a better position to ameliorate this
  problem. Using the present version of \ct it is possible to 
  specify the name of the eqdsk in the namelist of the very first file 
  (e.g. curray\_in) that is read (or the default name eqdsk.hb is used).
  Hence the eqdsk file can be opened
 at that point, the size of the eqdsk extracted, and then the
 arrays can be allocated dynamically at run time. All of this
 can be done at the very start of the program in rfdrive.f90.


 Limits of arrays in the curray\_in namelist inputs must be given
in the documentation. For example, what is the maximum allowed number
of ion species nspec? One may get the mistaken impression that dynamic
arrays are used to accommodate user input. For most variables that is
not the case however. Also there are subtleties involved in the input of the
antenna spectrum. We find that most users (including ourselves) have insufficient
experience to supply this data without further help from  a more
detailed description of the required input variables.

 Certain
features to improve the smoothness such as ray launch randomization using a  Gaussian 
ray generator or other appropriate means should be
included. Importance sampling could be used to better map out regions
of high absorption. In Fig.~\ref{Figbwav1a}(a), for example, we find
spikes in the ion heating and current drive profiles. Such features
become particularly troublesome when non-inductive current drive
studies are performed using a transport code driver.
  
There is redundancy of information across input files. For example,  
trxpl.out and curray\_in have variables \texttt{nprim, nimp, nspec}, etc.; either
the redundancy should be removed or a clear indication of precedence
rules must be given in the input instructions.
Some switches, e.g. \texttt{icurdr}, actually are used to create files
 (e.g. adjin). These switches and the name of the files  they create
 should be spelled out in the input instructions.
Finally assignment of Fortran io  unit numbers in the namelist input
  file does not seem reasonable (its purpose is not clear).



  When used within a transport code the definition of the driven
  current as some sort of flux surface averaged quantity is
  important  for accurate current drive calculations. The definition
  of this quantity needs to be made precise in \ct. The electron
  driven current in currayout is incorrectly labelled as Ma. The
  actual value given  is in amps/watt of injected power. 
   We should point out that for the DIII-D discharges
   presented above  \ct is
   quite sensitive to the initial launch conditions employed for the
   rays (in addition to the edge reflections problem).
   We found that both the number of poloidal starting points per
   toroidal refractive index, \texttt{nthin}, and the radial starting location
   of the ray in terms of poloidal flux, \texttt{psi0}, influenced the results
   significantly. We would recommend that features are added to the
   code to take care of these issues, perhaps by doing a preliminary
   scan to set parameters if time dependent transport run is to be done.
 Most of the cases run encountered some sort of (non-fatal)  error
 during execution. For example ``no solution of dispersion close to
   raypoint'' is one such error but there are at least several others.
 It is not reasonable to expect a non-specialist user to cope with such
 errors. We need to have an automatic
   assessment and fix for such failures in the code. A code such as \ct
   which is
   intelligent enough to trap these errors can most likely be made to
   take remedial action as well (e.g. perhaps a  more robust
but more time consuming  non-linear method
   must be invoked when an error condition is sensed).
  Relatively simple errors such as  ``starting point too close to the
 p-cutoff'' or ``absorbed power too small''  should be eliminated
  altogether since checking for such errors during a transport run is
  not feasible.
 
 Although not
explicitly shown in this report we did scan the effect that different
equilibrium file grid sizes have on \ct. The ray tracing is found  to
be slightly different in such cases but the effect is not great for
DIII-D provided reasonable grid sizes are used (equilibrium 65x65 or greater).
However we did encounter situations where changing eqdsk size caused 
\ct to fail. We did not determine the specific cause of these failures. 



 \subsection{\ct input}\label{ct_input}
 %\chapter {\ct input}
  In this appendix we present the actual antenna spectrum and other
  curray related inputs  used for the 
  cases presented in this review.  Note that 60, 83, and 117 MHz cases
 are included. The correct case is selected by setting  the second
 index equal to 1, 2, or 3 respectively.
 \tiny
\begin{verbatim}

  !---------------------start curray specific input ----------------------

  ! curray_path = 'SOME PATH TO YOUR FAVORITE CURRAY' DON'T SET TO USE DEFAULT
  save_curray_input = 0 ! =1 SAVES ALL CURRAY INPUT FILES
   psistep=0.025       !step in psi for ray tracing
   pkexpnt=0.001
   igraph=1            ! 0 NO GRAPHS FROM CURRAY ,1 CREATE GRAPHICAL OUTPUT FILE
   iprint=0            ! -1 CONDENSED,0= NONE,1=NORMAL,2=MORE , 3 = WAY TOO MUCH
   incrt = 1           ! number of harmonics in ion  damping is 2*incrt +1 
  
  epserr=0.02          !ERROR LIMIT FOR DISPERSION RELATION
  epser1=0.004         !SECOND ERROR LIMIT FOR DISPERSION RELATION
  idcur=1              !1 ANALYTIC CUR DRIVE EFFICIENCY, 3 = ADJOINT CALC,2 = DONT USE THIS
  nminor=0             !NUMBER OF MINORIY SPECIES
  kalfa=0              !0 IGNORE FAST ION DAMPING (APPARENTLY RELATES TO SLOWING DOWN DISTIRIBUTION)


  !nspect=6       !NUMBER OF TOROIDAL WAVE NUMBER BINS- SET BY  POWERSRT
  nrayptrt=6000

  psi_startrt(1) =  3*0.985   !starting value of psi for rays
  thgrilrt(1)=6*1.0   !approximate central location of antenna in degrees
                      !w.r.t. outer equatorial plane
  powersrt(1,1)= 4.8830e-3, 1.4940e-2, 3.7920e-2, 1.4220e-1, 3.6820e-1, 2.21852e-1
  powersrt(1,2)= 2.2948e-1, 4.9065e-1, 2.0648e-1, 4.8111e-2, 1.8340e-2, 6.9367e-3
  powersrt(1,3)= 2.2948e-1, 4.9065e-1, 2.0648e-1, 4.8111e-2, 1.8340e-2, 6.9367e-3

  anzinfrt(1,1)=    13.488,     8.093,     2.298,    -2.360,    -4.721,    -7.081
  anzinfrt(1,2)=   -2.1937,   -4.3875,   -6.5813,    2.4376,    7.3127,   11.7001
  anzinfrt(1,3)=   -1.5563,   -3.1126,   -4.6689,    1.7292,    5.1876,    8.3000

  anzsuprt(1,1)=    13.489,     8.094,     2.299,    -2.359,    -4.720,    -7.080
  anzsuprt(1,2)=   -2.1938,   -4.3876,   -6.5814,    2.4375,    7.3126,   11.7000
  anzsuprt(1,3)=   -1.5562,   -3.1125,   -4.6688,    1.7293,    5.1877,    8.3001

  nnkparrt(1,1)=   6*1
  nnkparrt(1,2)=   6*1
  nnkparrt(1,3)=   6*1

  !NUMBER OF N POLOIDALS IN EACH BIN:
  nnkpolrt(1,1)=         1,         1,         1,         1,         1,         1
  nnkpolrt(1,2)=         1,         1,         1,         1,         1,         1
  nnkpolrt(1,3)=         1,         1,         1,         1,         1,         1

  !UPPER LIMIT POLOIDAL REFRACTIVE INDEX:
  anpsuprt(1,1)= 6*-1.730
  anpsuprt(1,2)= 6*-1.250
  anpsuprt(1,3)= 6*-0.887
 
  !LOWER LIMIT POLOIDAL REFRACTIVE INDEX:
  anpinfrt(1,1)= 6*-1.731
  anpinfrt(1,2)= 6*-1.251
  anpinfrt(1,3)= 6*-0.887 


   nthinrt(1) =    11,    11,    11               ! NUMBER OF STARTING LOCATIONS 
   maxrefrt(1) =    6,    10,    10               ! MAX NUMBER EDGE REFLECTIOSN
   islofart(1) =   -1,    -1,    -1               ! -1 FAST WAVE, 1 SLOW WAVE
   heightrt(1) = 120.,  120.,  120.               ! ANTENNA HEIGHT , CM


   indvar=1     ! 1 USE TOTAL PHASE AS INDEP VARIABLE
   ichois=2     ! 1 HIGH FREQ DISP. , 2 NO FREQ LIMIT
   modcd=0      ! 0 EHSR KARNEY MODEL, 1 CHIU-KARNEY-MAU MODEL FOR J,P CALCS
   igrill=-3    ! -3 READ IN SPECTRUM, -1 ANALYTIC SPECTRUM
   bmaxrt=30    ! max Bessel function order
   idmpsw=1     ! 0 NO ION DAMPING, ` MAGNETIZED ION DAMPING,
                ! 2 UNMAGNETIZED ION DAMPING
   irayiort=0   ! details of data output to file rayiop
   beam_spec =-1,1,1,0 ! BEAM_SPEC(I) ,I =1 FULL ENERGY , 2 =HALF ,3 = THIRD 4 = FAST ALPHA
                      ! =1 MEANS TREAT THIS COMPONENT OF THE FAST IONS AS A SEPARATE SPECIES
                      ! =0 MEANS OMIT THIS COMPONENT. USED FOR EACH INJECTOR SEPARATELY
                      ! THUS TO GET CUR DRIVE DUE TO FAST ALPHAS ONLY
                      ! SET BEAM_SPEC = 0,0,0,1 FOR EXAMPLE.
                      ! TO USE ONLY THE FIRST ENERGY COMPONENT (OF ALL INJECTORS) SET
                      ! BEAM_SPEC = 1,0,0,0 , ETC. 
                      ! DENSITIES AND 2/3 FAST ION STORED ENERGY
                      ! DENSITY DIVIDED BY FAST ION DENSITY IS OUTPUT TO TRXPL.OUT
                      ! NUMBER OF FAST SPECIES IS (NO. INJECTORS ) * (SUM FROM 1 TO 3 OF { (BEAMSPEC(I))})
                      ! PLUS BEAM_SPEC(4)
  !--------------------end curray specific input --------------------------------------

\end{verbatim}
\normalsize
