 \section{Inone guidleines}
 Many people encounter difficulty in setting up the namelist inputs
 for \ot. These guidelines  may ease the pain a bit. Because namelist
 input is processed as a single block user code has no access to the
 individual lines in a namelist. This means that user code cant check
 for input errors. Instead we must rely on the compiler to generate
 suitable error messages. Unfortunately different compilers react
 differently to  namelist errors. The most user friendly is HP F90, which gives some useful
     information when namelist read errors are encountered. LF95
     and PGF90 are not forgiveing and basically tell you nothing.
     So one way to discover nameis input errors is to try and run 
     the code on HP machines.
     But due to compiler issues the  Onetwo
     is not up to date on HP . so if you are using new input
     parameters (as of Feb 2004) you cant use this option.
     
 \ot uses three namelists, called ``namelis1'',''namelis2'' and ``namelis3''.
 These three namelists must appear in a file called inone. Note that
 using scripts such as run12 might give you the impression that \ot
 reads the name of the namelist file from the cammand line. This is
 not the case however. Onetwo always looks for a file called inone in
 the current working directory. The inone file will be scanned from
 top to bottom until a namelist sentinel and the name of the namelist
 are encountered together in the form, for example, ``&namelis3''.
 Her the ``&'' is the namelist sentinel. Scanning stops when another 
 namelist sentinel is ecountered. This means that (a) the order of the
 namelists in inone is arbitrary,(b) outside of each individual
 namelist there can be arbitrary lines of code, and (c) namelists can
 be repeated. However if a namelsit is repeated it is the first
 occurance of the namelist that is used, NOT the last. Scanning stops
 and control is returned to the user code once the termination
 sentinel in the namelist is encountered.

 \begin{itemize}
   \item First of all you should be aware that although you input the
     data in file inone the code reads a file called INONE (all caps).
     This is the standard way to get rid of comments in namelist
     files. INONE is a copy of inone with all comments stripped out.
     Error messages put out by Onetwo refer to INONE not inone.
     INONE is destroyed after a succesfull run so you may never see
     it. The primary output file produced by \ot,called ``outone'',
     has a full listing of inone at its top so you can alwys tell what
     input was used to generate the output.

    \item Namelist sentinels are not uniform across compilers.
       LF95  will not accept the ``$'' namelist sentinel that is
   normaly used in our codes (PGF90 and HP F90 dont have this problem.)
   Hence to keep uniformity you should use ``&'' which is accepted by
   all compilers.
  \item Automatic generation of inone files using various tools 
    is a start. But none of these tools are  
    100% up to date or, as near as I can tell, bugfree.
    None of these tools were created or even verified by me and I know that
    there are some problems. Do not get
    confused by the fact that your automatically generated inone file
    doesnt run. The auto generated inone is only a skeleton.
    It alway needs manual modifications. Instead you may want to  start with
    an inone file that works and gradually morph it to the case you
    want to run. Doing it gradually means you know what set of lines
    caused the namelist to suddenly stop working. 
   \item As mentioned above there are three namelists in inone. Each namelist is
     processed independently. If you encounter an error the first
     thing to do is to determine what namelist the error occured in.
     This is most easily done  by setting the variable
     ``namelistvb = 1 '' in the first namelist (only) of inone.
     This will cause the message `` ... namelist X'' read to be
     printed to the screen if namelist number X was read succesfully. 
     Normally you will get 3 mesages, one for each namelist. The error
     occurs in that namelist for which  the message is not printed out.
   \item This is probably the most important point: Do not guess at
     what the input should be !  There is no reason to guess the form
     of an input variable. The file cray102.f contains a 100\% complete
      listing of all variables used in the namelist. Note that you may
      have to look at the namelists in cray102.f to see what namelist
      a variable belongs in 
 \item Last resort - If you still haven't found the problem there are
     several things you can do:
     \item You at least know what namelist the problem is in by using
     the namelistvb switch mentioned above.  Hence start by carefully
     perusing that namelist.
    \if you didnt pick up on anything (some problems are quite subtle
    and not readily visible) you can do a binary search on the
    offending  namelist : 
    A) go to the middle of the namelist and insert some garbage
    characters (make sure they are not comments)
    B) run the Onetwo code.
    C) Does the error message change as a result of the garbage
    characters?
      a) if not then move the garbage charcters up  in the namelist
         by about 1/2 the distance to the start.
      b) if the error message changed then you encountered the garbage
      characters before you encountered the true problem. Move the
      garbage characters down and try again.
    Obviously this is not a very satisfying approach but if you
   inadvertently get into this situation it may be the only way to
   fix the problem. (I have done this numerous times).

   \item Documentation is available in the form of the original
     report GAA-16178.  A  pdf document which concentrates on new
     features added since the original work is avaialble online.

 \end{itemize}
 \end{section}