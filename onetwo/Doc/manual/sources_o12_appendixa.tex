\section{An Example of TDEM Results}

The time dependent eqdsk mode of operation was discussed in the main part of the
text. Here we give some examples of the results obtained with this method for
the current drive scenario. The reader is reminded that the TDEM mode also
applies to (and in fact was constructed for) confinement analysis. The
application to current drive is quite subtle, however, so we present some 
details here. 

To begin with, Fig.~\ref{fig:TDEM1} shows the typical variation of the metric
parameters $ F,G,H $ and the parallel current density, \Eqref{jpar}.
Each curve in these figures represents a particular time during
the MHD evolution. The situation illustrated is for a negative
shear case (shot 87953). The curves were generated using the 
(fortran) \emph{MEPC} code (this code can easily be modified to generate
flux surface averages of various kinds so that the user does not
have to start from scratch if new data is required).

\begin{figure}
 \centering
 \includegraphics[height=.49\linewidth,angle=-90]{fcap.epsi}
 \includegraphics[width=.49\linewidth,angle=-90]{hcap.epsi}\\[.35in]
 \includegraphics[width=.49\linewidth,angle=-90]{gcap.epsi}
 \includegraphics[width=.49\linewidth,angle=-90]{curden.epsi}
 \caption{The metric parameters $F$, $G$, and $H$ and the parallel current
 density.}
 \label{fig:TDEM1}
\end{figure}

The variation in the metric parameters can have siginificant effects on the
transport derived quantities. For example in Fig.~\ref{fig:TDEMneut} we show the
neutron rate calculated using the TDEM mode and the result of the same
calculation when a single eqdsk is used.
\begin{figure}
 \centering
 \includegraphics[height=\textwidth,angle=-90]{neutron_rate.eps}
 \caption{Comparison of neutron production rates obtained using a single
 equilibrium slice or using a series of slices in TDEM
 mode.\label{fig:TDEMneut}}
\end{figure}

\begin{figure}
 \centering
 \includegraphics[width=\textwidth]{psi_time.eps}
 \caption{$\Psi$, the linear and a possible spline fit used for
 $\pdiff{\Psi}{t} $.  Shown is the magnetic axis value as a function of time.
Similar results hold for $\Psi $ at other locations.}
\end{figure}

\begin{figure}
 \centering
 \includegraphics[width=\linewidth]{dpsidt_time.eps}
 \caption{$\pdiff{\Psi}{t}$ at fixed $\rho$.\label{fig:TDEMdpsidt}}
\end{figure}

A number of beam heated shots ranging from strong negative shear (87953) to
weak negative shear (87937) to positive shear (89387,89388,89389) were examined
using this method.  The q-profiles are shown in Fig.~\ref{fig:TDEMq}.  The
object was to determine if this approach can be made to work in such a wide
variety of cases.
\begin{figure}
 \centering
 \includegraphics[angle=-90,width=\textwidth]{q_3casesxfig1.eps}
 \caption{q-profiles for various shots analyzed showing strong negative shear
 (87953), weak negative shear (87937), and positive
 shear(89387).\label{fig:TDEMq}}
\end{figure}

%--------------------------
As seen in Eqs. \eqref{eq:fday2_1} and \eqref{eq:fday2_4} only the product of
resistivity and ohmic current is determined by this approach. Consequently
additional information is required in order to  separate the product. In
quiescent discharges the resistivity should be neoclassical. The sensitivity  to
the form of the neoclassical model of resistivity that is used can be gauged by
comparing the Hinton and Hirshman models of the effective resistivity . As is
seen in Fig.~\ref{fig:TDEMweakstrong}, strong negative
central shear can be modeled about as well as the weak shear case. Both cases
deviate from the theoretical value of the ohmic current significantly, despite
the fact that a ``well behaved section'' of psi was used. 
\begin{figure}
 \centering
 \includegraphics[angle=90,width=.5\textwidth]{87937_1200.epsi}
 \includegraphics[angle=90,width=.49\textwidth]{87937_1475.epsi}\\
 \includegraphics[angle=90,width=.5\textwidth]{87953_1500.epsi}
 \includegraphics[angle=90,width=.49\textwidth]{87953_1700.epsi}
 \caption{Results for the weak shear case (top) and strong negative central
  shear case (bottom). \label{fig:TDEMweakstrong}}
\end{figure}

%-------------------------------
The comparison of experimental and theoretical current profiles for the positive shear cases is given in Fig.~\ref{fig:TDEMpositive}.
\begin{figure}
 \centering
 \includegraphics[angle=90,width=.5\textwidth]{89387ohm+others.epsi}
 \includegraphics[angle=90,width=.49\textwidth]{ohm_others89388.epsi}
 \caption{Results for the positive shear case.\label{fig:TDEMpositive}}
\end{figure}

If we use the experimental ohmic current together with the total parallel
current (which is derived from the pressure and poloidal current functions in
the MHD fit),  the non-inductive current may be  estimated. If we also assume
that the beam driven current is known, then an experimental determination of 
the bootstrap current becomes possible.  These currents are shown in
Fig.~\ref{TDEMboot}. 
\begin{figure}
 \centering
 \includegraphics[angle=90,width=.5\textwidth]{ohmic+total89389.epsi}
 \includegraphics[angle=90,width=.49\textwidth]{driven89389.epsi}
 \includegraphics[angle=90,width=.5\textwidth]{bootstrap89389.epsi}
 \caption{Ohmic, beam driven, and bootstrap current densities.\label{TDEMboot}}
\end{figure}
