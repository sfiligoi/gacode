  \documentclass[12pt]{report} 
   %% LaTeX2e file `abbrev.tex'
%% generated by the `filecontents' environment
%% from source `sources_o12' on 2010/05/19.
%%

   \newcommand{\vp}{ V^{\prime}}
   \newcommand{\ot}{Onetwo\xspace}
   \newcommand{\ct}{Curray\xspace}
   \newcommand{\pdiff}[2]{\frac{\partial{ #1}}{\partial{ #2}}}
   %useage \pdiff{#1}{#2}
   \newcommand{\ddiff}[1]{\frac{\partial}{\partial #1}}
   \newcommand{\strt}{\begin{eqnarray}}
   \newcommand{\trts}{\end{eqnarray}}
   \newcommand{\beq}{\begin{eqnarray}}
   \newcommand{\eeq}{\end{eqnarray}}
   \newcommand{\pdiffn}[3]{\frac{\partial^{#3}{#1}}{\partial{ #2}^{#3}}}
   \newcommand{\pdiffz}[3]{\frac{\partial{#1}}{\partial{ #2}}\bigg \vert_{#3}}
    \newcommand{\myint}{\int\limits_{\rho_{j-\frac{1}{2}}}^{\rho_{j+\frac{1}{2}}
}d\rho H \rho}
    \newcommand{\myind}{\int\limits_{\rho_{j-\frac{1}{2}}}^{\rho_{j+\frac{1}{2}}
}d\rho}
    \newcommand{\onehalf}{\frac{1}{2}}
    %\newcommand{\eqref}[1]{(\ref{#1})}
    \newcommand{\Eqref}[1]{Eq.~\eqref{#1}}
    \newcommand{\Eqrefs}[2]{Eqs.~\eqref{#1}-\eqref{#2}}

 \textheight8.5in
\newenvironment{narrow}[2]{%
   \begin{list}{}{%
   \setlength{\topsep}{0pt}%
   \setlength{\leftmargin}{#1}%
   \setlength{\rightmargin}{#2}%
   \setlength{\listparindent}{\parindent}%
   \setlength{\itemindent}{\parindent}%
   \setlength{\parsep}{\parskip}}%
\item[]}{\end{list}}

 % can keep my definitions is file abbrev.tex
   \usepackage{epsfig}
    \begin{document}
	\section{Effect of Charge Exchange}
        \Large \boldmath
        The fast ion distribution function depends on the factor
                      $ P_{cx}(v) $,defined as the 
	probability that a fast ion survives charge exchange at least down
        to speed v,having been born at speed $v_b$ . 

	 \beq
           P_{cx}(v)=  exp\left(-\tau_s \int_{v}^{v_b}\frac{v^2dv}
              {(v^3+v_c^3)\tau_{cx}}\right) \label{eq1}
         \eeq
        Here $ \tau_{cx} $ is the mean time time between charge exchange
	collisions,
	\beq 
	   \tau_{cx}=\frac{1}{n\sigma v}
	\eeq
	and is normally assumed to be large so that $P_{cx}\approx 1 $.
	A somewhat better aproximation assumes that $\tau_{cx} $ is a constant,
	evaluated at some particular value of $v=v_0 $so that eq \ref{eq1} .
	becomes
	\beq
	    P_{cx}(v)=exp\left(-\frac{\tau_s}{\tau_{cx}(v_0)}\right)
	\eeq
	If $v_0$ is taken as the average speed that a fast ion has
	during the slowing down process in the abscence of charge exchange,
	\beq
	    v_0=\frac{\int\frac{v^3dv}{v^3+v_c^3}}{\int\frac{v^2dv}{v^3+v_c^3}}
	\eeq
	and the Freeman-Jones cross section for charge exchange 
	shown in Fig {\ref{Figps}} is used, the resulting survival probability
	is given as the dashed line in Fig{\ref{Figps}}. Shown  
	in Fig{\ref{Figps}} as the solid line is the exact result for
        this case,which assumes that the neutral density ,n,is fixed
         at $1 \cdot 10^8 cm^{-3} $.
         From the figure we may conclude that at neutral densities as low
	as $1\cdot10^8 cm^{-3} $ it is important to 
        include a refined charge exchange factor in the fast ion
	distribution function .
	\begin{figure}[hbt] %note: figure environment not available in slides
        \centering
	\begin{tabular}{c@{\qquad what is it}c}
           \mbox{\epsfig{file=pic1.ps,height=9cm,width=8cm}}&
           \mbox{\epsfig{file=pic2.ps,height=9cm,width=8cm}} \\ 
            figure 1 &    figure 2  
	\end{tabular}
	\caption{(a)NRL (solid) and Bosh and Hale (dashed) charge
           exchange
             cross sections. Corresponding probabilities 
            against charge exchange (b)  }\label{Figps}
        \end{figure}
   \section{Input Description}

      Effect of charge exchange of fast ions with neutrals during the slowing
  down process is controlled by the integer valued switch $ifcx$ as follows: 

   \begin{tabular}{p{3cm}p{14cm}}
     $ ifcx \ge 0$ &          include the charge exchange effect in the fast 
	   ion distribution function in any volume element for which the local
	      mean time against charge exchange,$\tau_{cx}$, is related to the
	      local slowing down time,$\tau_s$,by
	       \beq
	         \tau_{cx} \leq  (ifcx)\tau_s
	       \eeq
	To always include charge exchange set $ifcx$ to a large number .
	(In the code any input value of  $ifcx \ge  ifcxmax (=10)$ means
	always (ie in all volume elements) include charge exchange. Note
	that setting $ifcx$ equal to 0 or a negative number means that
	charge exchange will be ignored.\\
     $ifcxmax$ &     a parameter 
	             that controls  \\
     $ennmin $ &     minimum neutral density,below which charge exchange will
                     be neglected .\\
   \end{tabular}

        ennmin if the local neutral density is less than
          $ ennmin $ then the effect of charge exchange will 
	($ifcxmax $ can be changed by a recompile of the code only)         


	\begin{figure}[hbt] %note: figure environment not available in slides
        \centering
           \mbox{\epsfig{file=pict.ddthermal,height=12cm,width=12cm}}
	\caption{Reaction Rates,$\frac{cm^3}{sec}$ . Solid line is the
	         new Bosch-Hale rate,dashed line is the old NRL rate}
        \label{Figps2}
        \end{figure}


    \end{document}

