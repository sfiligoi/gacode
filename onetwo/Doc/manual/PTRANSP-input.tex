% doc-input.tex
%---:----1----:----2----:----3----:----4----:----5----:----6----:----7-c
\documentstyle{article}
\headheight 0pt  \headsep 0pt  \topmargin 0pt  \oddsidemargin 0pt
\textheight 9.0in  \textwidth 6.5in
\renewcommand{\arraystretch}{2.0}
\begin{document}
\begin{center} 
{\bf {\tt doc-input.tex} \\
PTRANSP Input Documentation} \\
\vspace{1pc} Glenn Bateman \\
Princeton Plasma Physics Laboratory \\ Princeton, NJ, USA \\
22 June 1995
\end{center}

The PTRANSP code (currently named {\tt ptr}) 
is a time-dependent predictive transport post
processor to the TRANSP code.  It can also be run as a stand-alone
transport code.

The user controls the PTRANSP code through the namelist input in file
{\tt input}.  This namelist input has the following general form:
\begin{verbatim}
 $npt
 
 variable = value,  ! comment
 
 array = value1, value2, value3,  ! comments
 
 string = 'character string',  ! comments
 
 ! you can put as many comments as you want after "!"
 
 $end
\end{verbatim}

Output from PTRANSP appears in the following files:
\begin{verbatim}

output          ASCII file of long printouts
<runid>TF.PLN   ASCII file describing RPLOT output variables
<runid>NF.PLN   binary 1-D RPLOT output
<runid>MF.PLN   binary 2-D RPLOT output

\end{verbatim}
Here, {\tt <runid>} is the 8-character string given by the namelist
input variable {\tt crunid}.


\section{Profiles in PTRANSP}

There are three kinds of profiles in PTRANSP:
\begin{itemize}
\item Input profiles, which may be read in from RPLOT or prescribed
analytically.
\item Generalized density profiles, which are advanced by the transport
equations.
\item Derived profiles, which are used in computing the transport
coefficients and are sent to the output files.
\end{itemize}

The following rules apply to these three kinds of profiles:
\begin{itemize}
\item  Not all the input profiles need to be advanced by the transport
equations.
\item  Each profile that is advanced by the transport equations has to
have a corresponding input profile to provide
(1)  the initial profile and
(2)  boundary conditions (which may be placed anywhere within the
profile).
\item  At the present time, each profile that is advanced by the
transport equations has to have an input source - sink profile as a
function of time --- which may be read in from RPLOT or be prescribed by
analytical profiles (or the sum of both).
\item  Basic sets of derived profiles are constructed by first looking
for the profiles that are advanced by the transport equations, then
looking for profiles that are prescribed analytically, and finally,
looking for profiles that are read in from RPLOT, in that order.
That is, if some profile is read in from RPLOT and advanced by the
transport equations, the corresponding derived profile that appears in
the output will be the profile advanced by the transport equations.
If some profile were read in from RPLOT and also prescribed
analytically, but not advanced by the transport equations, then the
analytically prescribed profile would appear in the output.
\item  Some additional derived profiles, such as the electron density
$ n_e $ or all of the densities on zone boundaries, are then computed
from the basic sets of derived profiles.

\end{itemize}


\section{Specifying Input Profiles}

Each input profile, with index {\tt jn}, is controlled by the following
identifiers:
\begin{verbatim}

ntype(jn) =  the type of generalized density 
  = -1  for temperature (or energy per particle)
  =  0  to signal the end of the list of profiles
  =  1  for charged particle density
  =  2  for energy density
             
nuclearz(jn) = charge on the nucleus
  = -1  for electrons
  =  0  use the variable charge(jn) instead of nuclearz(jn)
            (for ions representing an average over several species)
  =  1  for hydrogenic
  =  2  for helium, ...

nisotope(jn) = isotope number
  =  0  use the variable aimass(jn) instead of nisotope(jn)
            (for ions representing an average over several species)
  =  1  for hydrogen
  =  2  for deuterium
  =  3  for tritium (if nuclearz(jn) = 1) 
          or He^3 (if nuclearz(jn) = 2),...

aimass(jn) = ion mass in AMU (for now, always specify aimass(jn)

charge(jn) = ion charge (default = nuclearz(jn))

charge2(jn) = ion charge^2
      Note:  in the case of composit ions (such as an impurity ion
      species that may be composed of several different ionization
      states, take the density weighted average of the ion charge^2
      Hence, charge2(jn) may not always be the same as (charge(jn))^2

nadvance(jn) = how the density is advanced by the transport equations
  =  0  to omit from the transport equations
          (may still be used to compute derived profiles, as needed.)
  =  1  to advance generalized density with the transport equations
  =  2  to include in the transport equations but force the profile
          to be given by the input profile as a function of time.
          (This is to allow off-diagonal elements of the transport
          matrix to act on these profiles.)

nprfgrid(jn)  type of grid for profile
  =  1  for zone centers xc(jr)
  =  2  for zone boundaries xb(jr)

nsrcgrid(jn)  type of grid for source
  =  1  for zone centers xc(jr)
  =  2  for zone boundaries xb(jr)

cabrprof(ja,jn) abbreviation for each density read in from RPLOT

convprof(ja,jn) conversion factor for each density

nrpoprof(ja,jn) operation performed on RPLOT readin of profile
  =  0  to start with a fresh profile 
  =  1  to add profile to rplotden(jp,jn)  (default) 
  =  2  to multiply profile with rplotden(jp,jn)

lrplprof(jn) = .true. (or .false.) to use (or not use) RPLOT profile,


cabrsorc(ja,jn) abbreviation for each source read in from RPLOT

convsorc(ja,jn) conversion factor for each source

nrpopsrc(ja,jn) operation performed on RPLOT readin of source

lrplsorc(jn) = .true. (or .false.) to use (or not use) RPLOT
                   sources and sinks,


lanaprof(jn) = .true. (or .false.) to use (or not use) analytic profile,

lintprof(jn) = .true. (or .false.) to use (or not use) interpolation
                   profile,

lanasorc(jn) = .true. (or .false.) to use (or not use) analytic profile,
                   sources and sinks,

lintsorc(jn) = .true. (or .false.) to use (or not use) interpolation
                   sources and sinks,

\end{verbatim}

Now, suppose a given profile is to be read in from RPLOT.
Then the user needs to specify the RPLOT
abbreviations for that profile ({\tt cabrprof(ja,jn)}) and the
conversion factors for that profile ({\tt convprof(ja,jn)}).
If the sources and sinks are to be read in from RPLOT for that profile,
then the user needs to specify the RPLOT
abbreviations for that source or sink ({\tt cabrsrc(ja,jn)}) and the
conversion factors for that source or sink ({\tt convsrc(ja,jn)}).
These profiles and sources and sinks can be constructed from sums of
RPLOT arrays (ja = 1, 2, \ldots).
Sinks can be subtraced from sources by using negative conversion
factors, as needed.

Note that the reason for requiring the user to give all the RPLOT
abbreviations in the namelist input is that the list of RPLOT
abbreviations has changed over time and may be different for different
installations.  This system provides flexibility.

All the switches {\tt lrplprof(jn)} through {\tt lintsorc(jn)} are true
by default.  If you leave these switches turned on, all the input
profiles will be computed as the sum of RPLOT, analytic,
and interpolated profiles.  
If no input were given for a given analytic profile (ie, no central or
edge values, for example) then that contribution to the profile would
be zero.  Computing a zero contribution to a profile does no harm.
The true or false switches are there make the code run more efficiently
or to make it more easy to turn off contributions at will.




Consider the following namelist input examples:
\begin{verbatim}

 ntype(1) = 1,        ! charged particle density
 nuclearz(1) = 1,
 nisotope(1) = 2,     ! deuterium
 ngrid(1) = 1,        ! zone centers
 cabrprof(1) = 'ND',  ! the only RPLOT abbreviation is 'ND'
 convprof(1,1) = 1.e6,  ! convert from (cm)^{-3} to m^{-3}
 cabrsrc(1,1)  = 'SVD', 'SWD',
                      ! add the volume and wall sources of deuterium
 convsrc(1,1)  = 1.e6, 1.e6,   ! source conversion factors
 nadvance(1) = 1,     ! advance deuterium density in transport eqns

 ntype(2) = -1,       ! temperature
 nuclearz(2) = -1,    ! for electrons
 ngrid(1) = 1,        ! zone centers
 cabrprof(1,2) = 'TE',  ! RPLOT abbreviation
 convprof(1,2) = 1.e-3, ! convert from eV to keV
    !  Note:  electron temperature is used to compute other profiles 

 ntype(3) = -1,       ! temperature
 nuclearz(3) = 0,     ! for all ions
 ngrid(1) = 1,        ! zone centers
 cabrprof(1,3) = 'TI',  ! RPLOT abbreviation
 convprof(1,3) = 1.e-3, ! convert from eV to keV
    !  Note:  ion temperature is used to compute other profiles 

 ntype(4) = 2,        ! electron energy density
 nuclearz(4) = -1,
 ngrid(1) = 1,        ! zone centers
 cabrsrc(1,4) = 'POH', 'PBE', 'QIE', 'PRAD', 'PION',
                ! electron heating - radiation - ionization losses
 convsrc(1,4) = 1.e6,  1.e6,  1.e6,  -1.e6,  -1.e6,
 nadvance(4) = 1,     ! advance in transport eqns
 
\end{verbatim}

Alternatively, suppose a given profile is to be specified by some
analytic profile.  Then, the user needs to specify the type of analytic
expression to be used, the central and edge values of the profile, and
parameters to give the shape of the profile.
Analytically prescribed profiles can be added to any profile that has
been read in from RPLOT.

For example,
\begin{verbatim}

 ntype(5) = 1,        ! charged particle density
 nuclearz(5) = 1,
 nisotope(5) = 1,     ! hydrogen
 ngrid(1) = 1,        ! zone centers
 timeden(1,5) = 0.0,   3.0,   7.0,     ! breakpoint times
 dencent(1,5) = 4.e19, 4.e19, 5.6e19,  ! central density
 denedge(1,5) = 3.e18, 5.e18, 7.4e18,  ! edge density
 denpinn(1,5) = 3*2.0,  ! inner exponent (ie, parabola to a power)
 denpout(1,5) = 3*1.6,  ! outer exponent (1-x^2)^(1.6)
 timesrc(1,5) = ...     ! breakpoint times for source function
 srcnode(1,1,5) = ...   ! nodes used to interpolate source
 nadvance(5) = 1,       ! advance in transport eqns
 
\end{verbatim}

\subsection{Analytically Prescribed Profiles}

The type of analytically prescribed profile is controlled by lprofil(j):
\begin{verbatim}
lprofil(1) = 0   for parabola raised to a power
lprofil(1) = -1  for cosine profile
lprofil(1) = -2  for zeroth order Bessel function
\end{verbatim}

The shapes of the analytically prescribed profiles are controlled by
\begin{verbatim}
\end{verbatim}


\subsection{Analytically Prescribed Sources}

The type of analytically prescribed source - sink 
is controlled by lprofil(j):
\begin{verbatim}
lsource(1) = 0   for parabola raised to a power
lsource(1) = -1  for cosine profile
lsource(1) = -2  for zeroth order Bessel function
\end{verbatim}

The shapes of the analytically prescribed sources - sinks 
are controlled by
\begin{verbatim}
\end{verbatim}

\section{RPLOT Input}

If {\tt lprofile(1) > 0}, profiles are read in from RPLOT files.
The names of all the profiles (ie, the RPLOT abbreviations) and the
conversion factors have to be given in the namelist input file.
For any given set of RPLOT files, you can find the names of the
available profiles by examining the {\tt <runid>TF.PLN} file or by
using RPLOT to list the names of the 2-D profiles.

The following is an example of the namelist input needed to read RPLOT
files:

\begin{verbatim}

 lprofile(1) = 1,  ! type of prescribed profile

 crpdisk = ' ',
 crpdir  = 'TFTR_73268A30',  ! directory for RPLOT file
 crprunid = '73268A30',      ! runid     for RPLOT file

 cabrxc = 'X',   ! zone centered grid from RPLOT file
 cabrxb = 'XB',  ! zone boundary grid from RPLOT file

 ! read equilibrium harmonics from 2-D RPLOT file

 ceqrc0 = 'RMM00',
 ceqrcm = 'RMM01', 'RMM02', 'RMM03', 'RMM04', 'RMM05',
 ceqysm = 'YMM01', 'YMM02', 'YMM03', 'YMM04', 'YMM05',
 convequ = 0.01,  ! conversion factor for equilibrium dimesions

\end{verbatim}

There are some essential arrays that must be available for the PTRANSP
run to procede.  For example, the grid specification, the zeroth and
first harmonics of the equilibrium flux surface shapes, must be there. 
If these names are not found, the PTRANSP code will issue an abort
message and stop.  If other nonessential names are not found, the
PTRANSP code will simply issue a warning message in file {\tt output}
and will then proceed.


\section{Profiles Advanced by Transport Equations}

The profiles of generalized densities are advanced in time by the
transport equations in subroutines {\tt ptrandif}, {\tt ptfindif}, and
{\tt ptrisolv}, with particle and heat fluxes computed in subroutine
{\tt ptdiffus} and sources and sinks computed in subroutine {\tt
ptsource}.
By ``generalized density,'' I mean a charged particle density or an
energy density or a momentum density.

Each generalized density is labeled by three identifiers:
\begin{verbatim}

nutype(jn) = the type of density (charge particle, energy, ...)
             and how the density is advanced by the transport equations
             
nuclearz(jn) = charge on the nucleus, or -1 for electrons

nuisotop(jn) = isotope number (0 for electrons)

\end{verbatim}

The following values can be used for {\tt nutype(jn)}:
\begin{verbatim}

nutype(jn) = -2 for prescribed profiles of energy density
           = -1 for prescirbed profiles of charged particle density
           =  0 end of list of profiles
           =  1 predict the profile of charged particle density
           =  2 predict the profile of energy density

\end{verbatim}

Consider the following example for setting up the generalized densities:
\begin{verbatim}

nutype(1) = 1,    ! predict the evolution of a charged particle density
nuclearz(1) = 1,  ! Hydrogenic species
nuisotop(1) = 2,  ! Deuterium

nutype(2) = 1,    ! predict the evolution of a charged particle density
nuclearz(2) = 6,  ! Carbon
nuisotop(2) = 12, ! isotope number

nutype(3) = 2,    ! predict the evolution of a thermal energy density
nuclearz(3) = 1,  ! ions (3 n_i T_i / 2 )
nuisotop(3) = 1,  ! (not used)

nutype(4) = 2,    ! predict the evolution of a thermal energy density
nuclearz(4) = -1, ! electrons (3 n_e T_e / 2 )
nuisotop(4) = 0,  ! (not used)

nutype(5) = -1,   ! prescribed charged particle density profiles  
nuclearz(5) = 1,  ! Hydrogenic species
nuisotop(5) = 1,  ! Hydrogen

nutype(6) = 0,    ! end of list of generalized density profiles

\end{verbatim}

These generalized densities can be given in any order and
the list of generalized densities can be restricted to any subset.
For example, if you wish to run only the ion thermal transport
equation, use the following lines in your dataset:
\begin{verbatim}

nutype(1) = 2,    ! predict the evolution of a thermal energy density
nuclearz(1) = 1,  ! ions (3 n_i T_i / 2 )
nuisotop(1) = 1,  ! (not used)

nutype(2) = 0,    ! end of list of generalized density profiles

\end{verbatim}






\section{Derived Profiles}

The derived profiles will use prescribed profiles (from RPLOT or
analytically prescribed) when the necessary generalized densities are
not available. 


\section{Output Frequency}

\begin{verbatim}
c
c..printout control:
c
c  nprlong      long printout every nprlong timesteps
c  fprlong      long printout every fprlong seconds
c  tprlong(jb)  long printout at times tprlong(jb)
c
c  nprshort     short printout every nprshort timesteps
c  fprshort     short printout every fprshort seconds
c  tprshort(jb) short printout at times tprshort(jb)
c
c..RPLOT output control
c
c  nrplot2d     2-D RPLOT output every nrplot2d timesteps
c  frplot2d     2-D RPLOT output every frplot2d seconds
c  trplot2d(jb) 2-D RPLOT output at times trplot2d(jb)
c
c  nrplot1d     1-D RPLOT output every nrplot1d timesteps
c  frplot1d     1-D RPLOT output every frplot1d seconds
c  trplot1d(jb) 1-D RPLOT output at times trplot1d(jb)

\end{verbatim}




\end{document}


\begin{verbatim}


\end{verbatim}
